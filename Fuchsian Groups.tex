\documentclass[10pt, twoside]{book}

%%%%%%%%%
% Maths %
%%%%%%%%%

\usepackage{math-fonts}
\usepackage{math-graphics}
\usepackage{math-symbols}
\usepackage{math-theorems}

%%%%%%%%%
% Title %
%%%%%%%%%

\title{Lecture Notes to Fuchsian Groups \\ \large{Winter 2020, Technion}}
\author{Lectures by Chen Meiri \\ \small{Typed by Elad Tzorani}}
\date{\today}

\begin{document}

\maketitle
\tableofcontents

\chapter{Preliminaries}

\section{The Hyperbolic Plane}

\subsection{The Riemann Sphere}

\begin{definition}[The Riemann Sphere]
The \emph{Riemann sphere} is a one-dimensional complex manifold, denoted $\hat{\mbb{C}} = \mbb{C} \cup \set{\infty}$, the charts of which are the following.
\begin{align*}
U_1 &= \prs{\mbb{C}, f_1} \\
U_2 &= \prs{\hat{\mbb{C}} \setminus \set{0}, f_2}
\end{align*}
where
\begin{align*}
f_1 \colon \mbb{C} &\to \mbb{C} \\
z &\mapsto z
\end{align*}
and
\begin{align*}
f_2 \colon \mbb{C} &\to \hat{\mbb{C}} \setminus \set{0} \\
z &\mapsto \frac{1}{z} \text{.}
\end{align*}
\end{definition}

\begin{definition}[Möbius Transformation]
A map $T \colon \hat{\mbb{C}} \to \hat{\mbb{C}}$ of the form
\[z \mapsto \frac{az + b}{cz + d}\]
where $ad - bc \neq 0$ is called a \emph{Möbius transformation}.
\end{definition}

\begin{notation}
\begin{enumerate}
\item We denote the image of $\pmat{a & b \\ c & d} \in \mrm{GL}_2\prs{\mbb{C}}$ in $\mrm{PGL}_2\prs{\mbb{C}}$ by $\bmat{a & b \\ c & d}$.
\item For every $g = \bmat{a & b \\ c & d} \in \mrm{PGL}_2\prs{\mbb{C}}$, we denote by $T_g$ the Möbius transformation $z \mapsto \frac{az + b}{cz + d}$.
\end{enumerate}
\end{notation}

\begin{lemma}
The set of Möbius transformations is a group under composition, and the map $g \mapsto T_g$ is an isomorphism between $\mrm{PGL}_2\prs{\mbb{C}}$ and the group of Möbius transformation.
\end{lemma}

\begin{proof}
It holds that
\begin{align*}
T_{g_1} \circ T_{g_2} \prs{z} &=  \frac{a_1 \prs{\frac{a_2 z + b_2}{c_2 z + d_2}} + 1}{c_1 \prs{\frac{a_2 z + b_2}{c_2 z + d_2}} + d_1}
\\&= \frac{\prs{a_1 a_2 + b_1 c_2} z + \prs{a_1 b_2 + b_1 d_2}}{\prs{c_1 a_2 + d_1 c_2} z + \prs{c_1 b_2 + d_1 d_2}}
\\&= T_{g_1 g_2}\prs{z} \text{.}
\end{align*}
In particular, $T_{g^{-1}}$ is the inverse of $T_g$.

The rest of the proof is clear.
\end{proof}

\begin{definition}[Generalised Circle]
A generalised circle in $\mbb{C}$ is either an Euclidean circle or an Euclidean straight line.
\end{definition}

\begin{lemma}
Let $T \colon \hat{\mbb{C}} \to \hat{\mbb{C}}$ be a Möbius transformation. Then
\begin{enumerate}
\item $T$ is an endomorphism of $\hat{\mbb{C}}$.
\item $T$ is conformal.
\item $T$ sends generalised circles to generalised circles.
\end{enumerate}
\end{lemma}

\subsection{Models of the Hyperbolic Plane}

\begin{definition}[The Upper Half Plane Model for the Hyperbolic Plane]
\begin{enumerate}
\item As a set, define $\mbb{H} \ceq \set{z \in \mbb{C}}{\Im\prs{z} > 0}$.
\item Let $\gamma \colon \brs{0,1} \to \mbb{H}$ be a piecewise continuously differentiable path given by $\gamma\prs{t} = x\prs{t} + iy\prs{t}$ for real functions $x\prs{t}, y\prs{t}$.
The \emph{hyperbolic length} of $\gamma$ is given by
\[h\prs{\gamma} \ceq \int_0^1 \frac{\sqrt{\prs{\frac{\diff x}{\diff t}}^2 + \prs{\frac{\diff y}{\diff t}}^2}}{y\prs{t}} \diff t = \int_0^1 \frac{\abs{\frac{\diff \gamma}{\diff t}}}{y\prs{t}} \diff t \text{.}\]
\item The \emph{hyperbolic distance} $\rho\prs{z,w}$ between two points $z,w \in \mbb{H}$ is defined as $\inf_\gamma h\prs{\gamma}$ where the infimum is taken over all piecewise continuously differentiable paths $\gamma$ from $z$ to $w$.
\end{enumerate}
\end{definition}

\begin{remark}
$\mbb{H}$ is a Riemann surface where for every $z \in \mbb{H}$, the inner product of $T_z H$ is given by
\[\prs{\prs{x_1, y_1},\prs{x_2,y_2}} = \frac{x_1 x_2 + y_1 y_2}{\prs{\Im z}^2} \text{.}\]
In particular, Euclidean angles are equal to hyperbolic angles.
\end{remark}

\begin{definition}[The Disc Model for the Hyperbolic Plane]
\begin{enumerate}
\item As a set, define $\mbb{U} \ceq \set{z \in \mbb{C}}{\abs{z} < 1}$.
\item Let $\gamma \colon \brs{0,1} \to \mbb{U}$ be a piecewise continuously differentiable path. The \emph{hyperbolic length} of $\gamma$ is given by
\[h_u\prs{\gamma} \ceq \int_0^1 \frac{2 \abs{\frac{\diff \gamma}{\diff t}}}{1 - \abs{\gamma\prs{t}}^2} \diff t \text{.}\]

\item The \emph{hyperbolic distance} $\rho_u\prs{z,w}$ between $z,w \in \mbb{U}$ is defined to be $\inf_\gamma h\prs{\gamma}$ where the infimum is taken over all piecewise continuously differentiable paths from $z$ to $w$.
\end{enumerate}
\end{definition}

\begin{remark}
It is clear that hyperbolic circles around $0$ are exactly Euclidean circles around it (with a generally different radius).
\end{remark}

\begin{remark}
Rotations around $0$ are isometries in the disc model.
\end{remark}

\begin{lemma}
Let $\pi$ be the Möbius transformation defined by
\[\pi\prs{z} = \frac{iz + 1}{z + i} \text{.}\]
Then
\begin{enumerate}
\item $\pi$ is a bijection from $\mbb{H}$ to $\mbb{U}$.
\item For every piecewise continuously differentiable path $\gamma \colon \brs{0,1} \to \mbb{H}$,it holds that $h_u\prs{\pi\prs{\gamma}} = h\prs{\gamma}$.
In particular, $\pi$ is an isometry.
\end{enumerate}
\end{lemma}

\begin{proof}
\begin{enumerate}
\item It holds that
\begin{align*}
\pi\prs{-1} &= -1 \\
\pi\prs{0} &= -i \\
\pi\prs{1} = 1 \text{.}
\end{align*}
Since Möbius transformations send generalised circles to generalised circles we get that $\pi$ sends $\mbb{R}$ to the unit circle.
Since $\pi\prs{i} = 0$ and $\pi$ is a homeomorphism of the Riemann sphere, we get the result.

\item Let $\gamma \colon \brs{0,1} \to \mbb{H}$ be a piecewise continuously differentiable path. Denote $\psi = \pi^{-1}$ and $\delta = \pi\prs{\gamma}$.
Then
\[\psi\prs{z} = \frac{iz - 1}{-z + i} = \frac{\prs{iz - 1}{- \bar{z} - i}}{\prs{-z + i}{-\bar{z} - i}} = \frac{\prs{z + \bar{z}} + i \prs{1 - \abs{z}^2}}{\abs{-z + i}^2} \text{.}\]
So,
\[\Im\prs{\psi\prs{z}} = \frac{1 - \abs{z}^2}{\abs{-z + i}^2} \text{.}\]
Since
\[\frac{\diff \psi}{\diff z} = \frac{-2}{\prs{-z+i}^2} \text{,}\]
we get that
\begin{align*}
h\prs{\gamma} &= \int_0^1 \frac{\abs{\frac{\diff \gamma}{\diff t}}}{\Im\prs{\gamma\prs{t}}} \diff t
\\&=
\int_0^1 \frac{\abs{\frac{\diff \psi\prs{\delta}}{\diff t}}}{\Im \prs{\psi\prs{\delta\prs{t}}}} \diff t
\\&=
\int_0^1 \frac{\abs{\frac{\diff \psi}{\diff z} \prs{\delta\prs{t}} \frac{\diff \delta}{\diff t}}}{\Im \prs{\psi\prs{\delta\prs{t}}}}
\\&= \int_0^1 \frac{2 \abs{\frac{\diff \delta}{\diff t}}}{1 - \abs{\delta\prs{t}}^2} \diff t
\\&= h_u\prs{\delta} \text{.}
\end{align*}

\end{enumerate}
\end{proof}

\subsection{Isometries of the Hyperbolic Plane}

\begin{lemma}
For every $g \in \bmat{a & b \\ c & d} \in \mrm{PSL}_2\prs{\mbb{R}}$ it holds that
\[T_g\prs{\mbb{H}} \subseteq \mbb{H} \text{.}\]
\end{lemma}

\begin{proof}
It's enough to show the inclusion $T_g\prs{\mbb{H}} \subseteq \mbb{H}$ since then
\[T_{g^{-1}}\prs{\mbb{H}} = \prs{T_g}^{-1} \prs{\mbb{H}} \subseteq \mbb{H}\]
which implies $T_g\prs{\mbb{H}} \supseteq \mbb{H}$ by applying $T_g$.

Now, we have
\begin{align*}
T_g\prs{z} &= \frac{az + b}{cz + d}
\\&=
\frac{\prs{az + b}\prs{c \bar{z} + d}}{\abs{cz + d}^2}
\\&=
\frac{ac \abs{z}^2 + adz + bc \bar{z} + bg}{\abs{cz + d}^2} \text{.}
\end{align*}
Thus,
\begin{align*}
\Im\prs{T_g\prs{z}} &= \frac{T_g\prs{z} - \overline{T_g\prs{z}}}{2 i}
\\&=
\frac{\prs{ad - bc} z - \prs{ad - bc} \bar{z}}{2i \abs{cz + d}^2}
\\&\underset{ad - bc = 1}{=}
\frac{\Im\prs{z}}{\abs{cz + d}^2} \text{.}
\end{align*}

\end{proof}

This lemma allows us to identify $\mrm{PSL}_2\prs{\mbb{R}}$ as a subgroup of $\mrm{Sym}\prs{\mbb{H}}$. The next lemma shows that even more is true.

\begin{lemma}
$\mrm{PSL}_2\prs{\mbb{R}} \subseteq \mrm{Isom}\prs{\mbb{H}}$.
\end{lemma}

\begin{proof}
It's enough to show that for every $g \in \mrm{PSL}_2\prs{\mbb{R}}$ and every piecewise continuously differentiable path $\gamma$ it holds that $h\prs{\gamma} = h\prs{T_g\prs{\gamma}}$.
Denote $T = T_g$ and $\delta = T\prs{\gamma}$. Then
\begin{align*}
h\prs{\delta} &= \int_0^1 \frac{\abs{\frac{\diff \delta}{\diff t}}}{\Im\prs{\delta\prs{t}}} \diff t
\\&=
\int_0^1 \frac{\abs{\frac{\diff T}{\diff z} \prs{\gamma\prs{t}} \frac{\diff \gamma}{\diff t}}}{\Im\prs{\delta\prs{t}}} \diff t
\\&\underset{\star}{=}
\int_0^1 \frac{\abs{\frac{\diff \gamma}{\diff t}}}{\gamma\prs{t}} \diff t
\\&=
h\prs{\gamma}
\end{align*}
where $\star$ follows from
\begin{align*}
\Im\prs{T_g\prs{z}} &= \frac{\Im \prs{z}}{\abs{cz + d}^2} \oplus \frac{\diff T}{\diff z} \\&=
\frac{a\prs{cz + d} - c\prs{az + b}}{\prs{cz + d}^2}
\\&= \frac{1}{\prs{cz + d}^2} \text{.}
\end{align*}
\end{proof}

\begin{corollary}
$\mrm{Isom}\prs{\mbb{H}}$ acts transitively on $\mbb{H}$.
\end{corollary}

\begin{proof}
It's enough to show that for every $z \in \mbb{H}$ there's $g \in \mrm{PSL}_2\prs{\mbb{R}}$ such that $T_g\prs{z} = i$.

If $z = x + yi$, take $g = \pmat{\frac{1}{\sqrt{y}} & - \frac{x}{\sqrt{y}} \\ 0 & \frac{1}{\sqrt{y}}}$, then
\[T_g\prs{z} = \frac{1}{y} \prs{x + yi} - \frac{x}{y} = i \text{.}\]
\end{proof}

\begin{lemma}
Let $\pi \colon \mbb{H} \to \mbb{U}$ be the isometry $z \mapsto \frac{iz + 1}{z + i}$ which we defined previously.
Then
\[\set{\pi T_g \pi^{-1}}{g \in \mrm{PSL}_2\prs{\mbb{R}}} = \set{\pmat{r & s \\ \bar{r} & \bar{s}}}{\substack{r,s \in \mbb{C} \\ \abs{r}^2 - \abs{s}^2 = 1}} \text{.}\]
In particular, by taking $r = e^{i \theta}$ and $s = 0$ we see that the action of $\mrm{PSL}_2\prs{\mbb{R}}$ on $\mbb{U}$ contains all the rotations around $0$.
\end{lemma}

\begin{proof}
It holds that
\[\pmat{i & 1 \\ 1 & i} \pmat{a & b \\ c & d} \pmat{i & -1 \\ -1 & i} = \frac{1}{2} \pmat{\prs{a+d} + i\prs{b-c} & b + c + i\prs{a-d} \\ \prs{b+c} - i \prs{a+d} & \prs{a+d} - i\prs{b-c}} \text{.}\]
Now, $\prs{a+d, a-d, b+c, b-c}$ can be any $4$-tuple. Specifically, for every $r,s \in \mbb{C}$ we have $a,b,c,d \in \mbb{R}$ such that $\pi T_g \pi^{-1} = \pmat{r & s \\ \bar{s} & \bar{r}}$, and by the equality from the determinants we get that $\bmat{a & b \\ c & d } \in \mrm{PSL}_2\prs{\mbb{R}}$.
\end{proof}

\begin{corollary}
Let $z_1, z_2, w_1, w_2 \in \mbb{H}$ be such that $\rho\prs{z_1, w_1} = \rho\prs{z_2, w_2}$, then there exists $g \in \mrm{PSL}_2\prs{\mbb{R}}$ such that $T_g\prs{z_1} = z_2$ and $T_g\prs{w_1} = w_2$.
\end{corollary}

\begin{proof}
Since $\mrm{PSL}_2\prs{\mbb{R}}$ acts transitively on $\mbb{H}$ we can assume that $z_1 = z_2$ and show that $\mrm{Stab}\prs{z_1}$ acts transitively on $\set{w \in \mbb{H}}{\rho\prs{z_1, w} = \rho\prs{z_1, w_1}}$.
We already showed this in the disc model, in the case $z_1 = i$.
\end{proof}

\begin{definition}
Let $\prs{X,d}$ be a metric space.
\begin{enumerate}
\item Let $x,y \in X$. A path $\gamma \colon \brs{a,b} \to X$ which joins $x$ and $y$ is called a \emph{geodesic segment} if for every $a \leq t_1 \leq t_2 \leq b$ it holds that $\abs{t_2 - t_1} = d\prs{\gamma\prs{t_1}, \gamma\prs{t_2}}$.

\item A path $\gamma \colon \mbb{R} \to X$ is called a \emph{geodesic line} if for every $a < b$ it holds that $\left. \gamma \right|_{\brs{a,b}}$ is a geodesic segment.
\end{enumerate}
\end{definition}

\begin{remark}
Let $\gamma$ be a geodesic segment or line.
Then $\gamma$ is determined by the image of $\gamma$ up to a composition with an isometry of $\mrm{R}$. Thus, we can identify geodesic segments and lines with their image up to orientation.
\end{remark}

\begin{lemma}
Let $b > a > 0$ be real numbers. Then $\set{iy}{a \leq y \leq b}$ is the unique geodesic segment between $ia$ and $ib$ and $\set{iy}{y > 0}$ is the unique geodesic line through $ia$ and $ib$.
\end{lemma}

\begin{proof}
We begin with the first part of the lemma.
Let $\gamma \colon \brs{0,1} \to \mbb{H}$ be a piecewise continuously differentiable path joining $ia$ and $ib$. For $t \in \brs{0,1}$ denote $\gamma\prs{t} = x\prs{t} + iy\prs{t}$ where $x\prs{t},y\prs{t} \in \mbb{R}$.
Then
\begin{align*}
h\prs{\gamma} &= \int_0^1 \frac{\sqrt{\prs{\frac{\diff x}{\diff t}}^2 + \prs{\frac{\diff y}{\diff t}}^2}}{y\prs{t}} \diff t
\\&\underset{\star}{\geq}
\int_0^1 \frac{\abs{\frac{\diff y}{\diff t}}}{y\prs{t}} \diff t
\\&\geq
\int_0^1 \frac{\frac{\diff y}{\diff t}}{y\prs{t}} \diff t
\\&=
\ln\prs{\frac{b}{a}} \text{.}
\end{align*}
Thus, $\rho\prs{ia, ib} \geq \ln\prs{\frac{b}{a}}$. If $y\prs{t} = i\prs{\prs{b-a}t + a}$, the above inequalities are equalities so $\rho\prs{ia, ib} = \ln\prs{\frac{b}{a}}$. The inequality $\star$ is an equality if and only if $x\prs{t} = 0$ for all $t \in \brs{a,b}$. It follows that the unique geodesic segment between $a$ and $b$ is $\set{iy}{a \leq y \leq b}$.

Now, it is clear that $\set{iy}{y > 0}$ is a geodesic line which passes through $ia$ and $ib$.
We want to show it's unique.

Assume towards a contradiction that there exists a geodesic line $\ell$ between $ia$ and $ib$ which isn't the positive part of the $y$-axis.
Then there's $z = x + iy \in \ell$ for which $x \neq 0$ and $\rho\prs{z, ia} > \rho\prs{z, ib}$. By the previous lemma, there exists $g \in \mrm{PSL}_2\prs{\mbb{R}}$ such that $T_g\prs{ia} = ia$ and $T_g\prs{z} \in i \mbb{R}$. Since $T_g$ sends generalised circles to generalised circles, $T_g\prs{ib} \notin i \mbb{R}$. Indeed, otherwise the image of the segment between $ia$ and $ib$ would belong to $i \mbb{R}$, and since $T_g$ sends generalised circles to generalised circles, it would send $i\mbb{R}$ to itself.

We get that there exists a geodesic between $ia$ and $T_g\prs{z} = ic$ which is not contained in $i\mbb{R}$, and this is impossible.
\end{proof}

\begin{theorem}

\begin{enumerate}
\item
Every distinct points $z,w \in \mbb{H}$ are contained in a unique geodesic segment and a unique geodesic line.

\item The geodesics in $\mbb{H}$ are semicircles and lines orthogonal to the real axis.
\end{enumerate}
\end{theorem}

\begin{proof}
\begin{enumerate}
\item
For every $g \in \mrm{PSL}_2\prs{\mbb{R}}$ it holds that $T_g\prs{\mbb{R} \cup \set{\infty}} = \mbb{R} \cup \set{\infty}$. If $z,w \in \mbb{H}$, by a previous lemma there exists $g \in \mrm{PSL}_2\prs{\mbb{R}}$ such that $T_g\prs{z} = ia$ and $T_g\prs{w} = ib$ for some $a,b \in \mbb{R}_+$.
Thus, $T_g^{-1} \prs{\brs{ia, ib}}$ is the unique geodesic segment between $z$ and $w$.

\item This follows from the fact that Möbius circles are conformal, send generalised circles to generalised circles, and sends $\mbb{R} \cup \set{\infty}$ to itself.      
\end{enumerate}
\end{proof}

\begin{corollary}
The geodesic segment in $\mbb{U}$ are segments of straight lines through zero or arcs of circles which are orthogonal to the unit circles.
\end{corollary}

\begin{theorem}
Let $z,w \in \mbb{H}$. Then
\begin{align*}
\sinh\prs{\frac{1}{2} \rho\prs{z,w}} = \frac{\abs{z-w}}{2 \prs {\Im\prs{z} \Im\prs{w}}^{\frac{1}{2}}} \text{.}
\end{align*}
\end{theorem}

\begin{proof}
Since $\mrm{PSL}_2\prs{\mbb{R}} \subseteq \mrm{Isom}\prs{\mbb{H}}$, the left side of the equation is invariant under the action of $\mrm{PSL}_2\prs{\mbb{R}}$. We first show that the right side is also invariant.

It's clear that the right side is invariant under maps of the form $z \mapsto az + b$ for $a,b \in \mbb{R}$. Since $\mrm{PSL}_2\prs{\mbb{R}}$ (viewed as a group of Möbius transformations) is generated by maps of the forms
\begin{align*}
z &\mapsto az + b, \, a,b\in\mbb{R} \\
z &\mapsto - \frac{1}{z}
\end{align*}
it's enough to show that the right side is invariant under these maps.

The right side is indeed invariant under $\frac{1}{z}$ since
\begin{align*}
\frac{\abs{\frac{1}{z} - \frac{1}{w}}}{2 \prs{\Im\prs{\frac{1}{z}} \Im\prs{\frac{1}{w}}}^{\frac{1}{2}}}
&=
\frac{\abs{\frac{z-w}{zw}}}{2 \prs{\Im\prs{\frac{z}{\abs{z}^2}} \Im\prs{\frac{w}{\abs{w}^2}}}^{\frac{1}{2}}}
\\&=
\frac{\abs{z-w}}{2 \prs{\Im\prs{z}\Im\prs{w}}^{\frac{1}{2}}} \text{.}
\end{align*}

Since both sides of the equation are invariant under the action of $\mrm{PSL}_2\prs{\mbb{R}}$, it's enough to prove the equality for $z=i$ and $w = ir$ for some $r \in \mbb{R}_+$.
Indeed,
\begin{align*}
\sinh\prs{\frac{1}{2} \rho\prs{i, ir}} &=
\sinh\prs{\frac{1}{2} \abs{\ln r}}
\\&=
\frac{\abs{\sqrt{r} - \frac{1}{\sqrt{r}}}}{2}
\\&=
\frac{\abs{r-1}}{2 \sqrt{r}}
\\&=
\frac{\abs{i - ir}}{2 \prs{\Im\prs{i} \Im\prs{ir}}^{\frac{1}{2}}} \text{.}
\end{align*}
\end{proof}

\begin{corollary}
\begin{enumerate}
\item The hyperbolic topology is equal to the Euclidean topology.
\item $\mbb{H}$ is a complete metric space.
\end{enumerate}
\end{corollary}

\begin{proof}
\begin{enumerate}
\item Let $z \in \mbb{H}$. If $\abs{\Im\prs{z} - \Im\prs{w}} < \frac{1}{2} \Im\prs{z}$ then
\[\frac{\abs{z-w}}{\sqrt{6} \Im\prs{z}} \leq \sinh \prs{\frac{1}{2} \rho\prs{z,w}} \leq \frac{\abs{z-w}}{\sqrt{2} \Im\prs{z}} \text{.}\]
\item We show that $\mbb{U}$ is a complete metric space, which implies the result since there's an isometry between $\mbb{U}$ and $\mbb{H}$.
Let $z,w \in \mbb{U}$, we have
\begin{align*}\label{equation:H_is_complete}
\sinh^2 \prs{\frac{1}{2} \rho\prs{z,w}} = \frac{\abs{z-w}^2}{\prs{1-\abs{z}^2}\prs{z-\abs{w}^2}} \text{.}
\end{align*}
Let $\prs{z_n}_{n \in \mbb{N}}$ be a hyperbolic Cauchy sequence. Then it's bounded in the hyperbolic metric, and \eqref{equation:H_is_complete} implies that it does not have a limit point on the unit circle and so is contained in a compact subset of the unit circle.

The result follows since \eqref{equation:H_is_complete} implies that on such a subset the hyperbolic and Euclidean metric are Lipschitz equivalent. 

\end{enumerate}
\end{proof}

\begin{exercise}
Prove that if $z,w \in \mbb{U}$ then
\[\sinh^2\prs{\frac{1}{2}\rho\prs{z,w}} = \frac{\abs{z-w}^2}{\prs{1 - \abs{z}^2} \prs{1 - \abs{w}^2}} \text{.}\]
\end{exercise}

\begin{theorem}
Let $\tau \colon \mbb{H} \to \mbb{H}$ be $\tau\prs{z} = -\bar{z}$. Then $\mrm{Isom}\prs{\mbb{H}} = \mrm{PSL}_2\prs{\mbb{R}} \rtimes \trs{\tau}$.
In particular, $\mrm{PSL}_2\prs{\mbb{R}}$ is a normal index two subgroup of $\mrm{Isom}\prs{\mbb{H}}$.
\end{theorem}

\begin{proof}
Clearly, $\tau$ is of order two. Since every index two subgroup is normal, it is enough to prove that for every hyperbolic isometry $s \in \mrm{Isom}\prs{\mbb{H}}$ there exists $g \in \mrm{PSL}_2\prs{\mbb{R}}$ such that $sT_g$ is either the identity or $\tau$.

There's $g \in \mrm{PSL}_2\prs{\mbb{R}}$ such that $T_g\prs{i} = s^{-1}\prs{i}$ and $T_g\prs{2i} = s^{-2}\prs{2i}$ (since $\mrm{PSL}_2\prs{\mbb{R}}$ is $2$-transitive). Then
$s T_g\prs{i} = i$ and $sT_g\prs{2i} = 2i$. Since isometries send geodesics to geodesics, for every $t>0$ it holds that $s T_g\prs{t i} = ti$.

Denote
\begin{align*}
U_+ &\ceq \set{z \in \mbb{H}}{\Re\prs{z} > 0} \\
U_{-} &\ceq \set{z \in \mbb{H}}{\Re\prs{z} < 0}\text{.}
\end{align*}
Since $s T_g$ is continuous it follows that $s T_g \prs{U_+} \subseteq U_+$ or $s T_g\prs{U_+} \subseteq U_-$.
In the first case denote $R \ceq s T_g$, and in the second case denote $R \ceq \tau s T_g$. In either case, $R\prs{U_+} \subseteq U_+$.

In order to finish the proof, we want to show that $R = \id$. For every $t > 0 $ we have
\begin{align*}
\frac{\abs{it - w}}{2 \prs{t \Im\prs{w}}^{\frac{1}{2}}} &= \sinh\prs{\frac{1}{2} \rho\prs{it, w}}
\\&=
\sinh\prs{\frac{1}{2} \rho\prs{R\prs{it}, R\prs{w}}}
\\&=
\sinh\prs{\frac{1}{2} \rho\prs{it, R\prs{w}}}
\\&=
\frac{\abs{it - R\prs{w}}}{2 \prs{t \Im\prs{R\prs{w}}}^{\frac{1}{2}}} \text{.}
\end{align*}
So,
\[\abs{it - w}^2 \Im\prs{R\prs{w}} = \abs{it - R\prs{w}}^2 \Im\prs{w} \text{.}\]
This holds for every $t$, which implies together with
\[\Im\prs{R\prs{w}} = \lim_{t \to \infty} \frac{\abs{it - w}^2 \Im\prs{R\prs{w}}}{t^2}\]
that
\[\Im\prs{R\prs{w}} = \lim_{t \to \infty} \frac{\abs{it - R\prs{w}}^2 \cdot \Im\prs{w}}{t^2} = \Im\prs{w} \text{.}\]
Now, for every $t > 0$ we get
\[\abs{it - w} = \abs{it - R\prs{w}}\]
which implies
$w = R\prs{w}$
or
$w = - \overline{R\prs{w}}$.
The latter case is impossible since $R\prs{U_+} \subseteq  U_+$.
\end{proof}

\begin{corollary}
Every element of $\mrm{Isom}\prs{\mbb{H}}$ is either conformal or anti-conformal.

An element of $\mrm{Isom}\prs{\mbb{H}}$ is conformal if and only if it belongs to $\mrm{PSL}_2\prs{\mbb{R}}$.
\end{corollary}

\begin{definition}
Let $\hat{\mbb{C}}$ be the Riemann sphere. The cross ratio of distinct points $z_1, z_2, z_3, z_4 \in \hat{\mbb{C}}$ is
\[\prs{z_1, z_2 : z_3, z_3} \ceq \frac{\prs{z_1 - z_2}{z_3 - z_4}}{\prs{z_2 - z_3}\prs{z_4 - z_1}} \text{.}\]
\end{definition}

\begin{lemma}
Möbius transformations preserve the cross ratio.
\end{lemma}

\begin{proof}
We prove this when $z_1, z_2, z_3, z_4 \in \mbb{C} \setminus \set{0}$. The other cases are left as exercise.

It is clear that maps of the form $z \mapsto az + b$, with $a \neq 0$, preserve the cross-ratio. Thus it's enough to prove that the map $z \mapsto -\frac{1}{z}$ preserves the cross-ratio.
Indeed,
\begin{align*}
\prs{z_1, z_2 ; z_3, z_4} &=
\frac{\prs{z_1 - z_2}{z_3 - z_4}}{\prs{z_2 - z_3}\prs{z_4 - z_1}}
\\&=
\frac{\prs{\frac{z_1 - z_2}{z_1 z_2}}{\frac{z_3 - z_4}{z_3 z_4}}}{\prs{\frac{z_2 - z_3}{z_2 z_3}}\prs{\frac{z_4 - z_1}{z_1 z_4}}}
\\&=
\frac{\prs{\frac{1}{z_1} - \frac{1}{z_2}}\prs{\frac{1}{z_3} - \frac{1}{z_4}}}{\prs{\frac{1}{z_2} - \frac{1}{z_3}}\prs{\frac{1}{z_4} - \frac{1}{z_1}}}
\\&=
\prs{\frac{1}{z_1}, \frac{1}{z_2} ; \frac{1}{z_3}, \frac{1}{z_4}} \text{.}
\end{align*}
\end{proof}

\begin{theorem}
Let $z,w \in \mbb{H}$ and let the geodesic joining $z,w$ have have end points $z^*$ and $w^*$ in $\mbb{R} \cup \set{\infty}$, chosen in a way that $z$ lies between $z^*$ and $w$. Then
\[\rho\prs{z,w} = \ln\prs{\prs{w, z^*; z, w^*}} \text{.}\]
\end{theorem}

\begin{proof}
Since both sides are invariant to the action of $\mrm{PSL}_2\prs{\mbb{R}}$ we can assume that $z = i$ and $w = ri$ with $r > 1$. Then $z^* = 0$ and $w^* = \infty$, so $r = \prs{w, z^*, z, w^*}$ and $\rho\prs{i,ir} = \ln\prs{r}$.
\end{proof}

\section{The Gauss-Bonnet Formula}

\begin{definition}[Hyperbolic Measure]
We define a measure $\mu$ on subsets of $\mbb{H}$ by
\[\mu\prs{A} = \int_A \frac{\diff x \diff y}{y^2}\]
for which this exists.
\end{definition}

\begin{theorem}
The hyperbolic area is invariant under $\mrm{PSL}_2\prs{\mbb{R}}$.
\end{theorem}

\begin{proof}
Let $f \colon \mbb{C} \to \mbb{C}$ given by
\[f\prs{x+iy} = u\prs{x,y} + i v\prs{x,y}\]
where $u,v \colon \mbb{C} \to \mbb{R}$.

By Cauchy-Riemann 
%TODO fill this in!
\begin{align*}
\frac{\del\prs{u,v}}{\del\prs{x,y}} &= \frac{\del u}{\del x} \frac{\del v}{\del y} - \frac{\del u}{\del y} \frac{\del v}{\del x}
\\&=
\prs{\frac{\del u}{\del x}}^2 + \prs{\frac{\del v}{\del y}}^2
\\&= \ldots
\end{align*}

Let $g = \bmat{a & b \\ c & d} \in \mrm{PSL}_2\prs{\mbb{R}}$. Recall that
\begin{align*}
\abs{\frac{\diff T_g}{\diff z}} &= \frac{1}{\abs{cz + d}^2} \\
\Im\prs{T_g\prs{z}} &= \frac{\Im \prs{z}}{\abs{cz + d}^2} \text{.}
\end{align*}
Then
\[T_g\prs{x + iy} = u\prs{x,y} + i v\prs{x,y}\]
so
\begin{align*}
\mu\prs{T_g\prs{A}} &= \int_{T_g\prs{A}} \frac{\diff u \diff v}{v^2}
\\&=
\int_A \frac{\del \prs{u,v}}{\del \prs{x,y}} \frac{\diff x \diff y}{v^2}
\\&=
\int_A \frac{1}{\abs{cz + d}^4 \cdot} \frac{\abs{cz + d}^4}{y \diff x \diff y}
\\&=
\mu\prs{A} \text{.}
\end{align*}
\end{proof}

\begin{definition}[$\tilde{\mbb{H}}$]
\begin{enumerate}
\item Define $\tilde{\mbb{H}} = \mbb{H} \cup \mbb{R} \cup \set{\infty}$.
\item A hyperbolic $n$-sided polygon is a closed subset of $\tilde{\mbb{H}}$ bounded by the closure of $n$ hyperbolic geodesic segments or rays.
\item A side of a polygon is the closure of a geodesic segment or ray which bounds to polygon.
\item A point $z \in \tilde{\mbb{H}}$ is called a vertex if it is the intersection of two distinct sides.
\end{enumerate}
\end{definition}

\begin{example}
There rare four types of hyperbolic triangles, which depend on the number of vertices on the boundary.
%TODO add image
\end{example}

\begin{theorem}[Gauss-Bonnet]
Let $\Delta$ be a hyperbolic triangle with angles $\alpha,\beta,\gamma$. Then
\[\mu\prs{\Gamma} = \pi - \alpha - \beta - \gamma \text{.}\]
\end{theorem}

\begin{proof}
First assume that $\Delta$ has a vertex on the boundary. Since $\mrm{PSL}_2\prs{\mbb{R}}$ preserves area, we may assume that this vertex is $\infty$.
Thus, two sides are given by equations $x = a$ and $x = b$ (and assume $a < b$).
By applying a transformation of the form
\[z \mapsto \lambda z + k\]
where $\lambda > 0$ and $k \in \mbb{R}$, we can assume that the third side of $\Gamma$ is an arc on the geodesic $\abs{z}^2 = 1$.

Pass segments from $0$ to the vertices of the triangle and call the angles between these and the real axis $\alpha$ and $\beta$.
Then
\begin{align*}
\mu\prs{\Delta} &=
\int_\Delta \frac{\diff x \diff y}{y^2}
\\&=
\int_a^b \diff x \int_{\sqrt{1 - x^2}}^\infty \frac{\diff y}{y^2}
\\&=
\int_a^b \frac{\diff x}{\sqrt{1 - x^2}}
\\&\underset{x = \cos\theta}{=}
\int_{\pi - \alpha}^\beta - \frac{\sin \theta}{\sin \theta} \diff \theta = \pi - \alpha - \beta - \cancelto{0}{\gamma} \text{.}
\end{align*}

In the other case, consider a triangle $\Delta = \mrm{ABC}$ with respective angles $\alpha,\beta,\gamma$. Continue the geodesic segment $\mrm{AB}$ to get an intersection $\mrm{D}$ with the boundary.
Let $\Delta' = \mrm{CBD}$ and $\Delta'' = \mrm{ABD}$.

%TODO add figure

Now, $\Delta'$ and $\Delta''$ have a vertex at infinity, so
\begin{align*}
\mu\prs{\Delta} &= \mu\prs{\Delta''} - \mu\prs{\Delta'}
\\&=
\pi - \prs{\alpha + \gamma + \theta} - \prs{\pi - \theta - \prs{\pi - \beta}}
\\&= \pi - \alpha - \beta - \gamma \text{.}
\end{align*}
\end{proof}

\section{Hyperbolic Geometry}

\begin{theorem}
Let $\Delta$ be a hyperbolic triangle with sides of hyperbolic lengths $a,b,c$ and opposite angles $\alpha,\beta,\gamma$.
Assume that $\alpha,\beta,\gamma > 0$ (so there is no vertex at the boundary).

The following holds.

\begin{description}
\item[The Sine Rule:]
\[\frac{\sinh\prs{a}}{\sin \alpha} = \frac{\sinh\prs{b}}{\sin \beta} = \frac{\sinh\prs{c}}{\sin \gamma}\]
\item[The First Cosine Rule:]
\[\cosh\prs{c} = \cosh\prs{a} \cosh\prs{b} - \cos\gamma \sinh\prs{a} \sinh\prs{b}\]
\item[The Second Cosine Rule:]
\[\cosh\prs{c} = \frac{\cos\alpha\cos\beta + \cos\gamma}{\sin \alpha \sin \beta} \text{.}\]
\end{description}
\end{theorem}

\begin{proof}
\begin{description}
\item[The First Cosine Rule:]
We use the disc model to prove the rule. Let $\Delta$ be a triangle in $\mbb{U}$ with sides $a,b,c$ and let $v_a, v_b, v_c$ be the vertices opposite to the respective sides.

We can assume $v_c = 0$ and $v_a = r \in \prs{0,1}$, and denote $v_b = z \in \mbb{U}$.
%TODO add fig 1.4
We have
\[\sinh^2 \prs{\frac{1}{2} \rho_u\prs{z,w}} = \frac{\abs{z-w}}{\prs{1-\abs{z}^2} \prs{1-\abs{w}^2}} \text{,}\]
but
\[\sinh^2\prs{\alpha} = \frac{1}{2} \cosh\prs{2 \alpha} - \frac{1}{2} \alpha\]
because
\begin{align*}
\prs{\frac{e^{\alpha} - e^{-\alpha}}{2}}^\alpha
&= \frac{e^{2 \alpha} - 2 + e^{-2 \alpha}}{4}
\\&= \frac{1}{2} \cdot \frac{e^{2 \alpha} + e^{-2 \alpha}}{2} - \frac{1}{2} \text{.}
\end{align*}
Hence
\[\cosh\prs{\rho_u\prs{z,w}} = \frac{2 \abs{z-w}}{\prs{1-\abs{z}^2} \prs{1-\abs{w}^2}} + 1 \text{.}\]
Then
\begin{align*}
\cosh\prs{a} &= \frac{1 + \abs{z}^2}{1 - \abs{z}^2} \\
\cosh\prs{b} &= \frac{1+r^2}{1-r^2} \\
\cosh\prs{c} &= \frac{2 \abs{z-r}^2}{\prs{1-\abs{z}^2}\prs{1-r^2}} + 1 \text{.}
\end{align*}

Using
\begin{align*}
\sinh\prs{a} &= \sqrt{\cosh^2\prs{\abs{z}} - 1} = \frac{2 \abs{z}}{1 - \abs{z}^2} \\
\sinh\prs{b} &= \sqrt{\cosh^2\prs{r} - 1} = \frac{2r}{1-r^2}
\end{align*}
and the Euclidean cosine rule
\[\cos \gamma = \frac{r^2 + \abs{z}^2 - \abs{2-r}^2}{2r \abs{z}}\]
we get
\begin{align*}
\cosh\prs{a} \cosh\prs{b} - \sinh\prs{a} \sinh\prs{b} \cos\prs{\gamma}
&=
\prs{\frac{1 + \abs{z}^2}{1-\abs{z}^2}} \prs{\frac{1 + r^2}{1-r^2}} - \frac{4r \abs{z}}{\prs{1-r^2}\prs{1-\abs{z}^2}} \cdot \frac{r^2 + \abs{z}^2 - \abs{z-r}^2}{2 r \abs{z}}
\\&=
\frac{\prs{1+r^2}\prs{1+\abs{z}^2} - 2r^2 - 2\abs{z}^2 + 2\abs{z-r}^2}{\prs{1-r^2}\prs{1-\abs{z}^2}}
\\&=
1 + \frac{2 \abs{z-r}^2}{\prs{1-r}^2 \prs{1-\abs{z}^2}}
\\&=
\cosh\prs{c} \text{.}
\end{align*}

\item[The Sine Rule:]
It holds by the first cosine rule that that
\begin{align*}
\prs{\frac{\sinh c}{\sin \gamma}}^2 &=
\frac{\sinh^2 c}{1 - \prs{\frac{\cosh a \cosh b - \cosh c}{\sinh \prs{a} \sinh\prs{b}}}^2}
\\&=
\prs{\cosh^2\prs{a} - 1}\prs{\cosh^2\prs{b} - 1} - \prs{\cosh\prs{a} \cosh\prs{b} - \cosh\prs{c}}^2
\\&= 1 + 2 \cosh\prs{a} \cosh\prs{b} \cosh\prs{c} - \cosh^2\prs{a} - \cosh^2\prs{b} - \cosh^2\prs{c}
\end{align*}
where the last term is symmetric in $a,b,c$.
\end{description}
\end{proof}

\chapter{Fuchsian Groups}

\section{Fuchsian Groups}

\subsection{Definitions}

\begin{definition}[$\mrm{SL}_2\prs{\mbb{R}}$]
Let $\mrm{SL}_2\prs{\mbb{R}}$ be the group of $2 \times 2$ real matrices with determinant $1$, with the topology from $\mbb{R}^4$ via $\pmat{a & b \\ c & d} \mapsto \pmat{a \\ b \\ c \\ d}$.
\end{definition}

Throughout the course, we endow $\mrm{PSL}_2\prs{\mbb{R}}$ with the quotient topology from $\mrm{GL}_2\prs{\mbb{R}}$.

We endow $\mrm{Isom}\prs{\mbb{H}}$ with the following topology.
Let $\tau \in \mrm{Isom}\prs{\mbb{H}} \setminus \mrm{PSL}_2\prs{\mbb{R}}$ and.
$U \subseteq \mrm{Isom}\prs{\mbb{H}}$ is open if and only if $U \cap \mrm{PSL}_2\prs{\mbb{R}}$ and $\tau U \cap \mrm{PSL}_2\prs{\mbb{R}}$ are open.

\begin{exercise}
\begin{enumerate}
\item
$\mrm{SL}_2\prs{\mbb{R}}, \mrm{PSL}_2\prs{\mbb{R}}, \mrm{Isom}\prs{\mbb{H}}$ are topological groups.
\item The actions of $\mrm{PSL}_2\prs{\mbb{R}}$ on $\mbb{H}$ and $\mbb{R} \cup \set{\infty}$ are continuous.
\end{enumerate}
\end{exercise}

\begin{definition}
Let
\[S\mbb{H} \ceq \set{\prs{z,\alpha}}{z \in \mbb{H}, \alpha \in \mbb{C}, \abs{\alpha} = \Im\prs{z}}\]
be the unit tangent bundle of $\mbb{H}$, which is homeomorphic to $\mbb{H} \times S^1$.
\end{definition}

\begin{definition}
For every $g \in \mrm{PSL}_2\prs{\mbb{R}}$ and $\prs{W,\alpha} \in S \mbb{H}$, denote $T_g \cdot \prs{w, \alpha} = \prs{T_g\prs{w}, D\prs{T_g}\prs{w}}$.
\end{definition}

\begin{definition}[Sharply Transitive Action]
A group action is called \emph{sharply transitive} if its transitive and the stabiliser of every element is trivial.
\end{definition}

\begin{lemma}
\begin{enumerate}
\item The map $\mrm{PSL}_2\prs{\mbb{R}} \times S\mbb{H} \to S\mbb{H}$ is a group action.
\item $\mrm{PSL}_2\prs{\mbb{R}}$ acts sharply transitive on $S\mbb{H}$.
\item The map $g \mapsto T_g\prs{\prs{i,i}}$ is a homeomorphism of $\mrm{PSL}_2\prs{\mbb{R}}$ and $S\mbb{H}$.
\end{enumerate}
\end{lemma}

\begin{proof}
\begin{enumerate}
\item Let $\prs{w,\alpha} \in S\mbb{H}$ and $g,h \in \mrm{PSL}_2\prs{\mbb{R}}$. We first show that $g \cdot \prs{w,\alpha} \in S\mbb{H}$.
It holds that
\begin{align*}
\Im\prs{T_g\prs{w}} &=
\abs{\frac{\diff T_g}{\diff z}\prs{w}} \cdot \Im\prs{w} \text{,}
\end{align*}
so
\begin{align*}
\abs{\left. D T_g \right|_{w \prs{\alpha}}} = \abs{\Im\prs{T_g\prs{w}}} \text{.}
\end{align*}

We now have to check that this is an action. It holds that
\begin{align*}
\prs{gh} \cdot \prs{w,\alpha} &= \prs{T_{gh}\prs{\alpha}, D T_{gh} \prs{w} \alpha}
\\&= \prs{T_g \prs{T_h\prs{w}}, D T_g \prs{T_h\prs{w}} \alpha}
\\&= g \cdot \prs{h \cdot \prs{w,\alpha}} \text{.}
\end{align*}

\item Let $\prs{w, \alpha} \in S \mbb{H}$. It is enough to show that there exists a unique $g \in \mrm{PSL}_2\prs{\mbb{R}}$ such that $g \cdot \prs{i,i} = \prs{w, \alpha}$. Recall that geodesic lines in $\mbb{H}$ are oriented generalised semicircles orthogonal to the real axis. Hence there exists a unique geodesic $\ell \colon \mbb{R} \to \mbb{H}$ which passes through $w$ and whose derivative at $w$ is $\alpha$.
Let $\gamma \colon \mbb{R} \to \mbb{H}$ be the geodesic given by $\gamma\prs{t} = ie^{t}$. Since $T_g$ sends geodesics to geodesics, it must send $i$ to $w$, send $\gamma$ to $\ell$, and respect the orientation of $\gamma$ and $\ell$. There exists a unique such $g$.

\item Prove this as an exercise.
\end{enumerate}
\end{proof}

%10.11.2020

We remind that $\mrm{PSL}_2\prs{\mbb{R}}$ is a topological group homeomorphic to $\mbb{H} \times S^1$. 

\begin{definition}[Fuchsian Group]
A subgroup of $\mrm{PSL}_2\prs{\mbb{R}}$ is called a \emph{Fuchsian group} if it is discrete.
\end{definition}

\begin{example}
$\mrm{PSL}_2\prs{\mbb{Z}}$ is a Fuchsian group.
\end{example}

\begin{definition}
Let $X$ be a metric space and let $G \leq \mrm{Isom}\prs{X}$.
\begin{enumerate}
\item A \emph{multiset} $M$ of subsets of $X$ is called locally finite if for every compact subset $K \subseteq X$, the multiset $\left[K \cap A \middle| A \in M\right]$ is finite.
\item We say that $G$ \emph{acts properly discontinuously} on $X$ if for every $x \in X$ the multiset $\left[ \set{gx} \middle| g \in G \right]$ is locally finite.
\end{enumerate}
\end{definition}

\begin{exercise}
Let $G$ be a group which acts on a metric space $X$ by isometries. Prove that the following conditions are equivalent.
\begin{enumerate}
\item%1
$G \acts X$ is properly discontinuous.
\item%2
Every $G$-orbit of $X$ is discrete, and the stabiliser of each point is finite.
\item For every sequence $\prs{g_n}_{n \in \mbb{N}} \subseteq G$ of distinct elements of $G$ and every $x \in X$ it holds that $\lim_{n\to\infty} g_n x \neq x$.
\item For every $x \in X$ there exists an open neighbourhood $V$ of $x$ such that the set $\set{g \in G}{gV \cap V \neq \ns}$ is finite.
\end{enumerate}
\end{exercise}

\begin{example}
$\mrm{PSL}_2\prs{\mbb{Z}}$ is discrete and acts continuously on $\del \mbb{H} = \mbb{R} \cup \set{\infty}$. The orbit of $0$ under $\mrm{PSL}_2\prs{\mbb{Z}}$ is $\mbb{Q}$ which is not a discrete subset. Hence the action is not properly discontinuous.
\end{example}

\begin{lemma}
For every $z \in \mbb{H}$ the stabiliser $\mrm{stab}_{\mrm{PSL}_2\prs{\mbb{R}}}\prs{z}$ is compact.
\end{lemma}

\begin{proof}
Since the action of $\mrm{PSL}_2\prs{\mbb{R}}$ on $\mbb{H}$ is continuous and transitive, it's enough to check the claim for a single point, say $z = i$.

Let $g = \bmat{a & b \\ c & d} \in \mrm{PSL}_2\prs{\mbb{R}}$. Assume
\[T_g\prs{i} = \frac{a_i + b}{c_i + d} = i \text{.}\]
Then
$ai + b = -c + di$ implies $a=d$ and $b=-c$. Then \[\mrm{stab}_{\mrm{PSL}_2\prs{\mbb{R}}}\prs{i} = \set{\bmat{a & b \\ -b & a}}{\substack{a^2 + b^2 = 1 \\ a,b \in \mbb{R}}} \text{.}\]
\end{proof}

\begin{lemma}
Let $w \in \mbb{H}$ and let $K \subseteq \mbb{H}$ be compact. Then
\[\set{g \in \mrm{PSL}_2\prs{\mbb{R}}}{T_g\prs{w} \in K}\]
is compact.
\end{lemma}

\begin{proof}
The hyperbolic and Euclidean topologies on $\mbb{H}$ are equal. Define a map
\begin{align*}
\rho \colon K &\to \mrm{PSL}_2\prs{\mbb{R}} \\
z &\mapsto g_z \ceq \bmat{a_z & b_z \\ 0 & a_z^{-1}}
\end{align*}
with
\begin{align*}
a_z &\ceq \sqrt{\frac{\Im\prs{z}}{\Im\prs{w}}} \\
b_z &\ceq -a_z^{-1} \Re\prs{z} - a_z \Re\prs{w} \text{.}
\end{align*}

For every $z \in K$
\begin{align*}
T_{g_z}\prs{w} &= a_z^2 w + b_z a_z w
\\&=
\frac{\Im\prs{z}}{\Im\prs{w}} w + \Re\prs{z} - \frac{\Im\prs{z}}{\Im\prs{w}} \Re\prs{w}
\\&= z \text{.}
\end{align*}
It's clear from definition that $\rho$ is continuous, so $M \ceq \im\prs{\rho}$ is compact. We get that
\[\set{g \in \mrm{PSL}_2\prs{\mbb{R}}}{gw \in K} = M \cdot \mrm{Stab}_{\mrm{PSL}_2\prs{\mbb{R}}}\prs{w}\]
where the last expression is the product of two compact subsets of $\mrm{PSL}_2\prs{\mbb{R}}$. A product of compact subsets of a topological group is compact, hence the result.
\end{proof}

\begin{theorem}
A subgroup $\Gamma \leq \mrm{PSL}_2\prs{\mbb{R}}$ is discrete if and only if it acts properly discontinuously on $\mbb{H}$.
\end{theorem}

\begin{proof}
\begin{itemize}
\item Assume first that $\Gamma$ is discrete. If $K \subseteq \mbb{H}$ is compact and $w \in \mbb{H}$, the previous lemma tells us that $\set{g \in \Gamma}{gw \in K}$ is the intersection of a discrete subset with a compact subset. Such an intersection is finite.

\item Assume that $\Gamma$ is not discrete. Then there exists a sequence $\prs{g_n}_{n \in \mbb{N}} \subseteq \Gamma$ of distinct elements, which converges to the identity.
Let $z \in \mbb{H}$. Then $\lim_{n\to\infty} g_nz = z$ so the $\Gamma$-action is not properly discontinuous.
\end{itemize}
\end{proof}

\begin{corollary}
Let $\Gamma \leq \mrm{PSL}_2\prs{\mbb{R}}$. Then $\Gamma$ is discrete if and only if every $\Gamma$-orbit is discrete.
\end{corollary}

\begin{proof}
The only if part is clear. For the other direction it is enough to show that if every orbit is discrete then the stabiliser of every element is finite.

Let $z \in \mbb{H}$, we know
$\mrm{Stab}_{\mrm{PSL}_2\prs{\mbb{R}}}\prs{z}$
is compact, so if $\mrm{Stab}_\Gamma\prs{z}$ is not finite, it is not discrete. Thus there exists a sequence $\prs{g_n}_{n \in \mbb{N}} \subseteq \mrm{Stab}_{\Gamma}\prs{z}$ of distinct elements, which converges to the identity.
Every element of $\mrm{PSL}_2\prs{\mbb{R}}$ stabilises at most one element of $\mbb{H}$. So, there exists $w \in \mbb{H}$ which is not fixed by any $g_n$. Since $g_n \xrightarrow{n\to\infty} \id$ it holds that $g_n w \xrightarrow{n\to\infty} w$ so the orbit of $w$ under $\Gamma$ is not discrete, which contradicts the assumption.
\end{proof}

\begin{definition}
Let $g \in \bmat{a & b \\ c & d} \in \mrm{PSL}_2\prs{\mbb{R}}$.
\begin{enumerate}
\item \emph{The trace of $g$} is $\tr\prs{g} = \abs{a+d}$.
\item $g$ is called \emph{elliptic} if $\tr\prs{g} < 2$.
\item $g$ is called \emph{parabolic} if $\tr\prs{g}=2$.
\item $g$ is called \emph{hyperbolic} if $\tr\prs{g} > 2$.
\end{enumerate}
\end{definition}

\begin{lemma} \label{lemma:conjugate_elements_in_subgroups_of_PSL_2}
\begin{enumerate}
\item If $g \in \mrm{PSL}_2\prs{\mbb{R}}$ is elliptic, it is conjugate to $\bmat{\cos \alpha & \sin \alpha \\ - \sin\alpha & \cos\alpha}$ for some $\alpha \in \mbb{R}$.

\item If $g \in \mrm{PSL}_2\prs{\mbb{R}}$ is parabolic, it is conjugate to $\bmat{1 & 1 \\ 0 & 1}$ or to $\bmat{1 & 1 \\ 0 & 1}^{-1}$.

\item If $g \in \mrm{PSL}_2\prs{\mbb{R}}$ is hyperbolic, it is conjugate to $\pmat{\lambda & 0 \\ 0 & \lambda^{-1}}$ for some $\lambda > 0$.
\end{enumerate}
\end{lemma}

\begin{remark}
If we have a discrete group $\Gamma$ with an elliptic/parabolic/hyperbolic element, we can assume that the element is of the form in \ref{lemma:conjugate_elements_in_subgroups_of_PSL_2} by conjugating $\Gamma$ by the appropriate elements of $\mrm{PSL}_2\prs{\mbb{R}}$.
\end{remark}

\begin{notation}
Let $g \in \mrm{PSL}_2\prs{\mbb{R}}$. We denote
\[\mrm{Fix}\prs{g} \ceq \set{z \in \tilde{\mbb{H}}}{T_g\prs{z} = z}\]
where $\tilde{\mbb{H}} \ceq \mbb{H} \cup \del \mbb{H}$.
\end{notation}

\begin{lemma}
\begin{enumerate}
\item If $g = \pmat{\cos \alpha & \sin \alpha \\ -\sin \alpha & \cos \alpha}$ for $\alpha \in \prs{0,\pi}$ then $\mrm{Fix}\prs{g} = i$.
\item
If $g = \bmat{1 & 1 \\ 0 & 1}$ then
$\mrm{Fix}\prs{g} = \set{\infty}$.
\item If $g = \bmat{\lambda & 0 \\ 0 & \lambda^{-1}}$ for $\lambda > 1$ then $\mrm{Fix}\prs{g} = \set{0, \infty}$. Also, $g$ preserves the geodesic $\set{iy}{y > 0}$.
\end{enumerate}
\end{lemma}

\begin{corollary}
\begin{enumerate}
\item If $g \in \mrm{PSL}_2\prs{\mbb{R}}$ is elliptic, $\mrm{Fix}\prs{g} = \set{z}$ is a unique point $z \in \mbb{H}$.
\item If $g \in \mrm{PSL}_2\prs{\mbb{R}}$ is parabolic, $\mrm{Fix}\prs{g} = \set{z}$ is a unique point $z \in \del \mbb{H}$.
\item If $g \in \mrm{PSL}_2\prs{\mbb{R}}$ is hyperbolic, then $\mrm{Fix}\prs{g} = \set{z,w}$ where $z \neq w$ and $z,w \in \del \mbb{H}$. Moreover, $g$ preserves the unique geodesic through $z$ and $w$. This geodesic is called \emph{the axis of $g$}. 
\end{enumerate}
\end{corollary}

\begin{exercise}
If $g \in \mrm{PSL}_2\prs{\mbb{R}}$ and $\trs{g}$ has an orbit of size $2$ in $\tilde{\mbb{H}}$, then $g$ is elliptic of order $2$.
\end{exercise}

\begin{solution}
Orbits under $\bmat{1 & 1 \\ 0 & 1}$ are orbits under $z \mapsto z+1$ which are infinite. So are orbits under $z \mapsto \lambda^2 z$.
Orbits under $\bmat{\cos \alpha & \sin \alpha \\ -\sin \alpha & \cos\alpha}$ can be of order $2$ if and only if $\alpha = \pm \pi$, from which $g$ is of order $2$.
\end{solution}

\begin{lemma}
Let $G \in \catname{Grp}$ and let $X \in G\mrm{-}\Set$. If $g,h \in G$ commute then
\[g\prs{\mrm{Fix}\prs{h}} = \mrm{Fix}\prs{h} \text{.}\]
\end{lemma}

\begin{proof}
Let $x \in \mrm{Fix}\prs{h}$. Then $h\prs{gx} = \prs{gh} x = gx$ so $gx \in \mrm{Fix}\prs{h}$. Hence
\[g \mrm{Fix}\prs{h} \subseteq \mrm{Fix}\prs{h} \text{.}\]
Similarly
\[g^{-1} \mrm{Fix}\prs{h} \subseteq \mrm{Fix}\prs{h}\]
by looking at $g^{-1}$, then
\[g\prs{\mrm{Fix}\prs{h}} = \mrm{Fix}\prs{h} \text{.}\]
\end{proof}

\begin{lemma}
The following hold, where isomorphisms are those of topological groups.
\begin{enumerate}
\item For every $\alpha \in \prs{0,\pi}$,
\[\mrm{Cent}_{\mrm{PSL}_2\prs{\mbb{R}}} \prs{\bmat{\cos \alpha & \sin \alpha \\ -\sin\alpha & \cos \alpha}} = \set{\bmat{\cos \beta & \sin \beta \\ -\sin \beta & \cos \beta}}{\beta \in \brs{0,\pi}} \cong S^1 \text{.}\]
\item It holds that
\[\mrm{Cent}_{\mrm{PSL}_2\prs{\mbb{R}}} \prs{\bmat{1 & 1 \\ 0 & 1}} = \set{\pmat{1 & a \\ 0 & 1}}{a \in \mbb{R}} \cong \prs{\mbb{R}, +} \text{.}\]
\item For every $\lambda > 0$ different than $1$ it holds that
\[\mrm{Cent}_{\mrm{PSL}_2\prs{\mbb{R}}} \prs{\bmat{\lambda & \\ & \lambda^{-1}}}{\eta > 0} \cong \prs{\mbb{R}, +} \text{,}\]
where the isomorphism is given by $\ln$.
\end{enumerate}
\end{lemma}

\begin{remark}
\begin{enumerate}
\item In particular it follows from the lemma that two non-identity elements $g, g'$ in $\mrm{PSL}_2\prs{\mbb{R}}$ commute if and only if $\mrm{Fix}\prs{g} = \mrm{Fix}\prs{g'}$.

\item If $\Gamma$ is a Fuchsian group, then the centraliser of every element $g \in \Gamma$ is cyclic and it is finite if and only if $g$ is elliptic.

\item An abelian Fuchsian group is cyclic.
\end{enumerate}
\end{remark}

\begin{example}
A Fuchsian group does not contain a subgroup isomorphic to $\mbb{Z} \times \mbb{Z}$.
\end{example}

\begin{lemma}
If $\Gamma \leq \mrm{PSL}_2\prs{\mbb{R}}$ is non-abelian then $\Gamma$ contains a non-elliptic element different from the identity.
\end{lemma}

\begin{proof}
We use the disc model $\mbb{U}$. Recall that we have an isomorphism $\rho \colon \mrm{Isom}\prs{\mbb{H}} \to \mrm{Isom}\prs{\mbb{U}}$ which is given by conjugation, so it preserves traces.

Let $g,h \in \Gamma$ be non-commuting elements. We show that if $g$ is elliptic then $\brs{g,h}$ is not elliptic. We can assume that
\[\rho\prs{g} \ceq \bmat{a & 0 \\ 0 & \bar{a}} \in \Gamma\]
and
\[\rho\prs{h} \ceq \bmat{b & c \\ \bar{c} & \bar{b}} \in \Gamma\]
where $\Im\prs{a} \neq 0$ and $c \neq 0$ since $g,h$ don't commute.
Then
\[\rho\prs{\brs{g,h}} = \bmat{\abs{b}^2 - a^2  \abs{c}^2 & *\\ * \abs{b}^2 - \bar{a}^2 \abs{c}^2} \text{.}\]
So
\begin{align*}
\tr\prs{\brs{g,h}} &=
\tr\prs{\rho\brs{g,h}}
\\&= 2 \abs{b}^2 - \prs{a^2 + \bar{a}^2}  \abs{c}^2
\\&= 2 \abs{b}^2 - \prs{\prs{a- a^{-1}}^2 + 2} \abs{c}^2
\\&= 2 \cancelto{1}{\prs{\abs{b}^2 - \abs{c}^2}} - \prs{a - \frac{1}{a}} \abs{c}^2
\\&= 2 + 4 \Im\prs{a}^2 \abs{c}^2
\\&> 2 \text{.}
\end{align*}
\end{proof}

\begin{theorem}
Let $\Gamma$ be a non-abelian Fuchsian groups. Then
\[N \ceq N_{\mrm{PSL}_2\prs{\mbb{R}}}\prs{\Gamma} \ceq \set{g \in \mrm{PSL}_2\prs{\mbb{R}}}{g \Gamma g^{-1} = \Gamma}\]
is a Fuchsian group.
\end{theorem}

\begin{proof}
Assume otherwise and let $\prs{h_n}_{n \in \mbb{N}}$ be a sequence of distinct elements of $N$ which converges to the identity.
Let $g \in \Gamma$. For every $n \in \mbb{N}$ it holds that $h_n g h_n^{-1} \in \Gamma$, and it holds that $\lim_{n\to\infty} h_n g h_n^{-1} = g$. Since $\Gamma$ is discrete, there exists $M_g$ such that for every $n > M_g$ it holds that $h_n g h_n^{-1} = g$. Then $h_n,g$ commute so $\mrm{Fix}\prs{g} = \mrm{Fix}\prs{h_n}$.
Since $\Gamma$ is not abelian, there exist $g_1,g_2 \in \Gamma \setminus \set{\id}$ which do not commute, so $\mrm{Fix}\prs{g_1} \neq \mrm{Fix}\prs{g_2}$.
On the other hand, for large enough $n \in \mbb{N}$ it holds that
\[\mrm{Fix}\prs{g_1} = \mrm{Fix}\prs{h_n} = \mrm{Fix}\prs{g_2} \text{,}\]
a contradiction.
\end{proof}

\subsection{Elementary Subgroups}

\begin{definition}[Elementary Subgroup]
A subgroup $\Gamma \leq \mrm{PSL}_2\prs{\mbb{R}}$ is called \emph{elementary} if it has a finite orbit in $\tilde{\mbb{H}}$.
\end{definition}

\begin{lemma}
A Fuchsian group is elementary if and only if it has a cyclic subgroup of index $2$.
\end{lemma}

\begin{proof}
\begin{itemize}
\item
Let $\Lambda \leq \Gamma$ of finite index. Then $\Lambda$ is elementary if and only if $\Gamma$ is. The if part is clear since abelian subgroups of $\mrm{PSL}_2\prs{\mbb{R}}$ have a fixed point.

Let $\Gamma$ be an elementary Fuchsian group. It is enough to prove that $\Gamma$ contains an index 2 abelian subroup since abelian Fuchsian groups are cyclic.
If all non-identity elements are elliptic, then $\Gamma$ is abelian. Assume $\Gamma$ contains a hyperbolic element $g$. Assume by conjugation that $\mrm{Fix}\prs{g} = \set{0,\infty}$. Then $\set{0,\infty}$ are the only finite orbits of $\trs{g}$ in $\tilde{\mbb{H}}$ so $\Gamma$ preserves $\set{0,\infty}$.

Let $h = \pmat{0 & 1 \\ -1 & 0}$, so that $T_h\prs{z} = -\frac{1}{z}$. Recall that $k \in \mrm{PSL}_2\prs{\mbb{R}}$ commutes with $g$ if and only if $\mrm{Fix}\prs{k} = \mrm{Fix}\prs{g}$.
Thus \[\Gamma \leq \mrm{Cent}_{\mrm{PSL}_2\prs{\mbb{R}}}\prs{g} \rtimes \trs{h} \text{.}\]
Indeed, $h$ is of order $2$ and conjugation by $h$ sends $g$ to $g^{-1}$.

It follows that $\Gamma \cap \mrm{Cent}_{\mrm{PSL}_2\prs{\mbb{R}}}\prs{g}$ is an abelian subgroup of $\Gamma$ of index at most $2$.

\item
Assume now that $\Gamma$ contains a parabolic element $g$ but no hyperbolic elements. $\trs{g}$ has a unique finite orbit in $\tilde{\mbb{H}}$ and it is $\mrm{Fix}\prs{g}$. Thus every element of $\Gamma$ fixes $\mrm{Fix}\prs{g} \subseteq \del \mbb{H}$, so every element is either either parabolic with the same fixed point, hyperbolic, or elliptic. It cannot be hyperbolic by assumption, and cannot be elliptic since it does not preserve a point in $\mbb{H}$.

We get that for non-identity elements $k \in \Gamma$ it holds that $\mrm{Fix}\prs{h} = \mrm{Fix}\prs{g}$ so $\Gamma$ is abelian.
\end{itemize}
\end{proof}

\begin{exercise}
Show that if $g,h \in \mrm{PSL}_2\prs{\mbb{R}}$, $g$ is hyperbolic, $\brs{g,h} \neq \id$ and $\trs{g,h}$ is a Fuchsian elementary group then $\trs{g,h} \cong \mbb{Z} \rtimes \mbb{Z}_2$ and $h\prs{\mrm{Fix}\prs{g}} = \mrm{Fix}\prs{g}$.
\end{exercise}

\begin{lemma}
Any non-elementary group $\Gamma \leq \mrm{PSL}_2\prs{\mbb{R}}$ contains a hyperbolic element.
\end{lemma}

\begin{proof}
Since $\Gamma$ is not abelian, it contains a non-elliptic element $g \in \Gamma \setminus \set{\id}$. We can assume by conjugation that $g = \bmat{1 & 1 \\ 0 & 1}$. Since $\Gamma$ is not abelian, it contains an element $h = \bmat{a & b \\ c & d} \in \mrm{PSL}_2\prs{\mbb{R}}$ which does not commute with $g$. Hence $c \neq 0$ or $b \neq 0$. But,
\begin{align*}
\tr\prs{g^n h} &= \tr\bmat{a + cn & b+dn\\c&d} = \abs{a + cn + d} \\
\tr\prs{h g^n} &= \tr\bmat{a & b+an \\ c & d+bn} = \abs{a + d + bn}
\end{align*}
where both terms go to $\infty$ as $n \to \infty$, so at some point $g^n h$ and $h g^n$ are hyperbolic.
\end{proof}

\begin{exercise}
Let $\Gamma \leq \mrm{PSL}_2\prs{\mbb{R}}$ be non-elementary and let $X \subseteq \del \mbb{H}$ be finite. Then there's a hyperbolic $g \in \Gamma$ such that $\mrm{Fix}\prs{g} \cap \Gamma = \ns$.
\end{exercise}

\begin{theorem}
Let $\Gamma \leq \mrm{PSL}_2\prs{\mbb{R}}$ be non-elementary and assume that $\Gamma$ does not contain an elliptic element. Then $\Gamma$ is discrete.
\end{theorem}

\begin{proof}
Let $\prs{g_n}_{n \in \mbb{N}}$ be a sequence of elements of $\Gamma$ which converges to $\id$. Denote $g_n = \bmat{a_n & b_n \\ c_n & d_n}$. We have to show that $g_n = \id$ for all $n$ large enough.
By the previous exercise, it is enough to show that for every hyperbolic $h \in \Gamma$, if $n$ is large enough (depending on $h$) it holds that $h, g_n$ have a common fixed point.

Let $h \in \Gamma$ be a hyperbolic element. We can assume that $h = \bmat{u & \\ & u^{-1}}$ for some $u > 1$. If $g = \bmat{a & b \\ c & d}$. Then
\[\brs{h,g} = \bmat{ad - bc u^2 & ab \prs{u^2 - 1} \\ cd \prs{u^{-2} - 1} & ad - bc u^{-2}} \text{.}\]
So,
\[\tr\prs{\brs{h,g}} = 2 \prs{ad - bc} - bc\prs{u - u^{-1}}^2 = 2 - bc \prs{u-u^{-1}}^2 \text{.}\]
Also
\begin{align*}
\tr\prs{\brs{h, \brs{h,g}}} &= 2 - abcd \prs{u^2 - 1}\prs{u^{-2} - 1}\prs{u - u^{-1}}
\\&=
2 + abcd \abs{\prs{u^2 - 1}\prs{u^{-2} - 1}\prs{u - u^{-1}}^2} \text{.}
\end{align*}
Adding these two equations gives that if $\tr\prs{\brs{h,g}} \geq 2$ and $\tr\prs{\brs{h,\brs{h,g}}} \geq 2$ then $bc \leq 0$ and either $bc = 0$ or $ad \leq 0$.
Applying this to the sequence $\prs{g_n}_{n \in \mbb{N}}$ and noting that $\lim_{n \to \infty} a_n d_n = 1$ for large enough $n$ we get $b_n = c_n = 0$. If $b_n = 0$ then $0$ is a fixed point of $g_n$ and if $c_n = 0$ then $\infty$ is a fixed point of $g_n$. In either case, $g_n$ and $h$ have a common fixed point.
\end{proof}

%17.11.2020

\begin{theorem}[Jorgensen Inequality]\label{theorem:Jorgensen_inequality}
Let $g,h \in \mrm{PSL}_2\prs{\mbb{R}}$ and assume that $\trs{g,h}$ is a non-elementary discrete group.
Then
\[\abs{\tr\prs{g}^2 - 4} + \abs{\tr\brs{g,h} - 2} \geq 1 \text{.}\]
\end{theorem}

\begin{theorem}
A non-elementary group $\Gamma \leq \mrm{PSL}_2\prs{\mbb{R}}$ is discrete if and only if for every $g,h \in \Gamma$ the group $\trs{g,h}$ is discrete.
\end{theorem}

\begin{theorem}
The only if part is clear. Assume therefore that $\trs{g,h}$ is discrete for every $g,h \in \Gamma$ and assume towards a contradiction that there's $\prs{g_n}_{n \in \mbb{N}} \subseteq \Gamma \setminus \set{\id}$ such that $g_n \xrightarrow{n\to\infty} \id$.

We proved that $\Gamma$ contains hyperbolic elements $h_1, h_2$ such that $\mrm{Fix}\prs{h_1} \cap \mrm{Fix}\prs{h_2} = \ns$.  The only element of $\mrm{PSL}_2\prs{\mbb{R}}$ which fixes $4$ points is the identity. Thus it is enough to prove that for every hyperbolic element $h$ there exists $M_h$ such that for every $n \geq M_h$ it holds that $\mrm{Fix}\prs{h} = \mrm{Fix}\prs{g_n}$.

Let $h$ be a hyperbolic element. Choose $M_h$ large enough such that for every $n \geq M_h$ it holds that
\[\abs{\tr\prs{g_n}^2 - 4} + \abs{\tr\brs{h,g_n} - 2} < 1\]
and the order of $g_n$ is not $2$.
By Jorgensen inequality, $\trs{g,h}$ is elementary. Since the only finite orbits of $\trs{h}$ are contained in $\mrm{Fix}\prs{h}$ we get that $\mrm{Fix}\prs{g} = g_n \prs{\mrm{Fix}\prs{h}}$. Since $\abs{\mrm{Fix}\prs{h}} = 2$, either $\mrm{Fix}\prs{g_n} = \mrm{Fix}\prs{h}$ or $g_n$ switches the two elements in $\mrm{Fix}\prs{h}$. The latter is impossible since the order of $g_n$ is not $2$, and thus there are no $\trs{g_n}$-orbits of size $2$.
\end{theorem}

\begin{lemma}\label{lemma:Jorgensen:lemma_1}
Let $g,h \in \mrm{PSL}_2\prs{\mbb{R}}$. Define $g_1 = g$ and for every $n \geq 1$ define $g_N \ceq g_{n-1} h g_{n-1}^{-1}$. If for some $n \geq 0$ it holds that $gh = h$, then $\trs{g,h}$ is elementary and $g_2 = h$.
\end{lemma}

\begin{proof}
The claim is clear if $g_0 = h$ so assume that $g_n = h$ for some $n \geq 1$. We claim that for every $k \in \brs{n}$ it holds that $\mrm{Fix}\prs{h} = \mrm{Fix}\prs{g_k}$. Indeed, assume that $k \in \brs{n}$ and $\mrm{Fix}\prs{h} = \mrm{Fix}\prs{g_k}$. The claim follows if $\abs{\mrm{Fix}\prs{h}} = 1$ since $\abs{\mrm{Fix}\prs{g_{k-1}}} = \abs{\mrm{Fix}\prs{h}}$ (since $g_{k-1}$ and $h$ are conjugate).

If $\abs{\mrm{Fix}\prs{h}} = 2$ then $h$ is hyperbolic so $g_{k-1} = g_{k-2} h g_{k-2}$ is hyperbolic and thus cannot switch the two points in $\mrm{Fix}\prs{h}$. We deduce that $\mrm{Fix}\prs{h} = \mrm{Fix}\prs{g_{k-1}}$ also in this case.

It follows that $h$ and $g_1$ have the same fixed points so they commute and $g_2 = g_1 h g_1^{-1} = h$.

Finally, \[\mrm{Fix}\prs{h} = \mrm{Fix}\prs{g_1} = g_0\prs{\mrm{Fix}\prs{h}} = g \prs{\mrm{Fix}\prs{h}}\]
so $\mrm{Fix}\prs{h}$ contains a $\trs{h,g}$-orbit, so $\trs{g,h}$ is elementary.
\end{proof}

\begin{lemma}\label{lemma:Jorgensen:lemma_2}
Let $g,h \in \mrm{PSL}_2\prs{\mbb{R}} \setminus \set{\id}$. Assume that $\abs{\tr\prs{g}^2 - 4} + \abs{\tr\brs{g,h} - 2} < 1$. Define $g_0 \ceq g$, and for every $n \geq 1$ define $g_n \ceq g_{n-1} h g_{n-1}^{-1}$. Then
\begin{enumerate}
\item If $h$ is parabolic, $\lim_{n \to \infty} g_n = h$.
\item If $h$ is hyperbolic $\lim_{n \to \infty} h^n g_{2n} h^{-n} = h$.
\item If $h$ is elliptic, $\lim_{n\to\infty} g_n = h$.
\item If $h$ is 
\end{enumerate}
\end{lemma}

\begin{proof}
Denote
\[g_n = \bmat{a_n & b_n \\ c_n & d_n} \text{.}\]
\begin{enumerate}
\item%1
We can assume that $h = \pmat{1 & 1 \\ 0 & 1}$. Then $\tr\prs{\brs{h,g}} = 2 + c_0^2$ and
\[g_{n+1} = g_n h g_n^{-1} = \bmat{1 - a_n c_n & a_n^2 \\ -c_n^{2} & 1 + a_n c_n} \text{.}\]
It follows that
\begin{enumerate}
\item $\abs{c_0} < 1$ since $\abs{c_0^2} = \abs{\tr\brs{h,g} - 2} < 1$.
We know $c_n = -\prs{c_0}^{2^n}$ and $\abs{a_{n+1}} \leq 1 + \abs{a_n}$. So $\abs{a_n} \leq n + \abs{a_0}$.
\item Since $\abs{c_0} < 1$ and $c_n = -\prs{c_0}^{2^n}$ it holds that $c_n \to 0$. By the bound on $a_n$ we get also $a_n c_n \to 0$. By the formula for $a_{n+1}$ we get $a_n \to 1$. Then $g_n \to h$.
\end{enumerate}
\item%2
We can assume that
\[h = \bmat{u & 0 \\ 0 & u^{-1}}\]
with $u > 1$. Then
\[\mu \ceq \abs{\tr\prs{g}^2 - 4} + \abs{\tr\brs{h,g} - 2} = \prs{1+\abs{bc}}\prs{u - u^{-1}}^2 < 1 \text{,}\]
and
\[g_{n+1} = g_n h g_n^{-1} = \bmat{a_n d_n u - b_n c_n u^{-1} & a_n b_n \prs{u^{-1} - u} \\ c_n d_N \prs{u - u^{-1}} & a_n d_n u^{-1} - b_n c_n u} \text{.}\]
We deduce the following.
\begin{enumerate}
\item
\[b_{n+1} c_{n+1} = a_n b_n c_n d_n \prs{u - u^{-1}} \prs{u^{-1} - u} = -b_n c_n \prs{1 + b_n c_n} \prs{u - \frac{1}{u}}^2 \text{,}\]
so \[\abs{b_n - c_n} \leq \mu^n \abs{b_0 c_0} \xrightarrow{n\to\infty} 0\text{.}\]
\item \[a_n d_n = 1 + b_n c_n \xrightarrow{n\to\infty} 1\]
so
$a_n \to u$ and $d_n \to u^{-1}$.
\item \[\abs{\frac{b_{n+1}}{b_n}} = \abs{a_n \prs{u - u^{-1}}} \xrightarrow{n\to\infty} \abs{u \prs{u-u^{-1}}} \leq \sqrt{\mu} \cdot \abs{u} \text{.}\]
Hence
\[\abs{\frac{b_{n+1}}{u^{n+1}}} \leq \sqrt{\mu} \abs{\frac{b_n}{u^n}}\]
so
$\frac{b_n}{\mu^n} \xrightarrow{n\to\infty} 0$.
\item Similarly, \[\abs{\frac{c_{n+1}}{c_n}} = \abs{d_n \prs{u - u^{-1}}} \xrightarrow{n\to\infty} \abs{u^{-1}\prs{u - u^{-1}}} \leq \sqrt{\mu} u^{-1} \leq \abs{c_{n+1} u^{n+1}} \leq \sqrt{\mu} \abs{c_n u^n} \text{,}\]
so $c_n u_N \xrightarrow{n\to\infty} 0$.
\item We get by the above parts that
\begin{align*}
h^n g_{2n} h^{-n} &= \bmat{a_{2n} & b_{2n} u^{-2}  \\ c_{2n} u^{2n} & d_n} \xrightarrow{n\to\infty} h \text{.}
\end{align*}
\end{enumerate}
\item%3
We use the disc model and regard $g, h \in \mrm{Isom}\prs{\mbb{U}}$. We can assume that $h = \bmat{u & 0 \\ 0 & u^{-1}}$ where $u \in \mbb{C}$ is such that $\abs{u} = 1$.

Let
\[\mu \ceq \abs{\tr\prs{g}^2 - 4} + \abs{\tr\brs{g,h} - 2} = \prs{1 + \abs{bc}} \prs{u - u^{-1}}^2 < 1\]
and let
\[g_{n+1} = g_n h g_n^{-1} = \bmat{a_n d_n u - b_n c_n u^{-1} & a_n b_n \prs{u^{-1} - u} \\ c_n d_N \prs{u - u^{-1}} & a_n d_n u^{-1} - b_n c_n u} \text{.}\]
as above.
We deduce that following.
\begin{enumerate}
\item%a
\[b_{n+1} c_{n+1} = - b_n c_n \prs{1 + b_n c_n} \prs{u - \frac{1}{u}}^2\]
so $\abs{b_n c_n} \leq \mu^n \abs{b_0 c_0}$
and $b_n c_n \xrightarrow{n\to\infty} 0$, as before.

\item%b
$a_n d_N = 1 + b_n c_n \xrightarrow{n\to\infty} 1$, so $a_n \xrightarrow{n\to\infty} u$ and $d_n \xrightarrow{n\to\infty} u^{-1}$.

\item%c
\[\abs{\frac{b_{n+1}}{b_n}} = \abs{a_n \prs{u - u^{-1}}} \xrightarrow{n\to\infty} \abs{u \prs{u - u^{-1}}} \leq \sqrt{\mu} \cdot \abs{u} = \sqrt{\mu} < 1\text{,}\]
so $b_n \xrightarrow{n \to\infty} 0$.
\item
\[\abs{\frac{c_{n+1}}{c_n}} = \abs{d_n \prs{u - u^{-1}}} \xrightarrow{n\to\infty} \abs{u^{-1} \prs{u -u^{-1}}} \leq \sqrt{\mu} \cdot \abs{u}^{-1} = \sqrt{\mu}\]
so $c_n \xrightarrow{n \to\infty} 0$.
\end{enumerate}
\end{enumerate}
\end{proof}

\begin{proof}[\ref{theorem:Jorgensen_inequality}]
Assume \[\abs{\tr\prs{g}^2 - 4} + \abs{\tr\brs{h,g} - 2} < 1 \text{.}\]
Define $g_0 = g$ and for every $n \geq 1$ assume $g_n = g_{n-1} h g_{n-1}^{-1}$.
We claim that there exists $n$ such that $g_n = h$. If this is true, \ref{lemma:Jorgensen:lemma_1} implies that $\trs{g,h}$ is elementary.

We prove our claim. If $h$ is parabolic or elliptic, this follows from \ref{lemma:Jorgensen:lemma_2} and discreteness.
If $g$ is hyperbolic, \ref{lemma:Jorgensen:lemma_2} and discreteness imply that that for large enough $n$ it holds that $h^n g_{2n} h^{-n} = h$ so $g_{2n} = h$.
\end{proof}

\subsection{Fundamental Domains}

\begin{definition}[Fundamental Set]
Let $G$ be a group and $X$ a $G$-set. A representative set for the $G$-orbits is called a \emph{fundamental set}.
\end{definition}

\begin{definition}[Fundamental Domain]
Let $\Gamma$ be a Fuchsian group. A subset $D \subseteq \mbb{H}$ is called a fundamental domain for $\Gamma$ if the following holds.
\begin{enumerate}
\item $D$ is a domain (i.e. connected \& open).
\item There is a fundamental set $F$ such that $D \subseteq F \subseteq \bar{D}$.
\item The hyperbolic area of $\del D$ is zero.
\end{enumerate}
\end{definition}

\begin{lemma}
Let $D$ be a fundamental domain.
If $z_1, z_2 \in \bar{D}$ are in the same $\Gamma$-orbit, than $z_1, z_2 \in \del D$.
\end{lemma}

\begin{proof}
Part 2 of the definition implies that at least one of $z_1, z_2$ belongs to $\del D$. Assume $z_1 \in D$. There exists a sequence $\prs{w_n}_{n \in \mbb{N}} \subseteq D$ which converges to $z_2$. If $g z_1 = z_2$ then $\prs{g^{-1} w_n}_{n \in \mbb{N}}$ converges to $z_1$. Since $D$ is open, for large enough $n$, $g^{-1} w_n \in D$. But, since \[z_1 = \lim_{n\to\infty} g^{-1} w_n \neq \lim_{n \to \infty} w_n = z_2\] we get that $g^{-1} w_n \neq w_n$ for large enough $n$. This contradicts part 2 of the definition.
\end{proof}

\begin{theorem}
Let $F_1, F_2$ be measurable fundamental sets for a Fuchsian group $\Gamma$. Then $\mrm{h-Area}\prs{F_1} = \mrm{h-Area}\prs{F_2}$.
\end{theorem}

\begin{proof}
Let $\mu$ denote the hyperbolic area.
We have
\begin{align*}
\mu\prs{F_1} &= \mu\prs{F_1 \cap \mbb{H}}
\\&=
\mu\prs{F_1 \cap \brs{\bigcup_{g \in \Gamma} g F_2}}
\\&=
\sum_{g \in \Gamma} \mu\prs{F_1 \cap g F_2}
\\&=
\sum_{g \in \Gamma} \mu\prs{g^{-1} F_1 \cap F_2}
\\&=
\sum_{g \in \Gamma} \mu\prs{g F_1 \cap F_2}
\\&=
\mu\prs{F_2} \text{.}
\end{align*}
\end{proof}

\begin{theorem}
Let $\Gamma \leq \mrm{PSL}_2\prs{\mbb{R}}$ be a Fuchsian group and let $\Lambda$ be a subgroup of $\Gamma$ of index $m$. If $\Gamma$ has a measurable fundamental set $F$, $\Lambda$ has a measurable fundamental set of measure $\mu\prs{F} \cdot m$.
\end{theorem}

\begin{proof}
Assume $\prs{g_i}_{i \in \brs{m}}$ are representatives to right cosets.
Then
\begin{align*}
\Gamma = \bigcup_{i \in \brs{m}} \Lambda g_i \text{.}
\end{align*}

If $w,z \in F$, $h \in \Lambda$, $i,j \in \brs{n}$ and $h g_1 z = g_2 w$ then
\[a_2^{-1} h g_1 z = w \text{.}\]
Since $F$ is a fundamental set for $\Gamma$ we get $z = w$ and $z \in \mrm{Fix}\prs{g_2^{-1} h g_1}$.

Clearly, $\bigcup_{i \in [m]} F g_i$ contains a fundamental set for $\Gamma$. Thus, there exists a fundamental set $E$ of $\Gamma$ such that
\[\bigsqcup_{i \in \brs{m}} g_i \prs{F \setminus \bigcup_{g \in \Gamma \setminus \set{1}} \mrm{Fix}\prs{g}} \subseteq E \subseteq \bigcup_{i \in \brs{m}} g_i F \text{.}\]
The sets $\mrm{Fix}\prs{g}$ for $g \in \mrm{\Gamma}\setminus\set{1}$ are countable, hence $E$ is measurable with $\mu\prs{E} = m \cdot \mu\prs{F}$.
\end{proof}

\begin{lemma}
Let $D$ be a fundamental domain for a Fuchsian group $\Gamma$. Denote by $\quot{D}{\Gamma}$ the quotient space after identifying points in the same orbit.
We get the following commutative diagram.

\[
\begin{tikzcd}
\bar{D} \arrow[r, "\tau"] \arrow[d, "\tilde{\pi}"] & \mbb{H} \arrow[d, "\pi"]
\\
\quot{\bar{D}}{\Gamma} \arrow[r, "\theta"] & \quot{\mbb{H}}{\Gamma}
\end{tikzcd}
\]

Then
\begin{enumerate}
\item $\theta,\tau$ are injective.
\item $\pi, \tilde{\pi}, \theta$ are surjective. Then, $\theta$ is bijective.
\item $\pi, \tilde{\pi}, \tau$ are continuous.
\item $\pi$ is open.
\end{enumerate}
\end{lemma}

\begin{proof}
Every part of the proof is clear except maybe that $\theta$ is continuous and that $\pi$ is open.
We now have the following.
\begin{itemize}
\item Recall that $V \subseteq \quot{\mbb{H}}{\Gamma}$ is open if and only if $\pi^{-1}\prs{V}$ is open. If $U \subseteq \mbb{H}$ is open,
\[\pi^{-1}\prs{\pi\prs{U}} = \bigcup_{g \in \Gamma} gU\]
is open as a union of open sets.

Hence $\pi$ is open.

\item Let $V \subseteq \quot{\mbb{H}}{\Gamma}$ be open. Then
\[\tilde{\pi}\prs{\prs{\tau^{-1}\prs{\pi^{-1}\prs{V}}} \cap \bar{D}} \text{.}\]
Denoting $U \ceq \prs{\tau^{-1}\prs{\pi^{-1}\prs{V}}} \cap \bar{D}$, this is open and
\[\tilde{\pi}^{-1}\prs{\tilde{\pi}\prs{U}} = U\]
is open, so $\tilde{\pi}\prs{U}$ is open.
\end{itemize}
\end{proof}

\subsection{Examples}

\begin{example}
Let $X = \mbb{C} \setminus \set{0}$ and $g \colon X \to X$ given by $g\prs{z} = 2z$. Then a fundamental domain for example is $\set{z \in \mbb{Z}}{\abs{z} \in \prs{1,2}}$. Then $\quot{\bar{D}}{\Gamma} \cong \mbb{T}^2$ is compact.

A different fundamental domain would be the same set with the part where $x,y \geq 1$ replaced by the domain bounded by $y = e^{-x}$ and $y = \frac{1}{2} e^{-x}$. In this case $\quot{\bar{D}}{\Gamma}$ is non-compact.
\end{example}

\begin{example}
Let $g,h \in \mrm{PSL}_2\prs{\mbb{Z}}$ be $g\prs{z} = 2z$ and $h\prs{z} = \frac{3z + 4}{2z + 3}$.
Consider the domains in the figure %TODO add reference
%TODO add fig. 1.6
Then
\begin{align*}
g\prs{\mbb{H} \setminus \bar{B}} &= E \\
g^{-1}\prs{\mbb{H} \setminus \bar{E}} &= B
\end{align*}
and
\begin{align*}
h\prs{\mbb{H} \setminus A} &= C \\
h^{-1}\prs{\mbb{H} \setminus \bar{C}} &= A \text{.}
\end{align*}
For every $k \in \Gamma \setminus \set{\id}$ where $\Gamma = \trs{g,h}$. We have $kw \notin D$.
In particular, $\Gamma w$ is discrete, so $\Gamma$ is discrete.

$D$ is a fundamental domain for $\Gamma$.
\begin{itemize}
\item We showed that every orbit intersects $D$ in at most one point.
\item Assume $z \notin \bar{D}$ so that $z \in A \cup B \cup C \cup E$.
\begin{itemize}
\item If $z \in B$, we have $\rho\prs{z,w} > \rho\prs{gz, w}$.
\item If $z \in C$, we have $\rho\prs{z,w} > \rho\prs{h^{-1} z, w}$.
\item If $z \in A$, we have $\rho\prs{z,w} > \rho\prs{hz, w}$.
\item If $z \in E$, we have $\rho\prs{z,w} > \rho\prs{g^{-1}z, w}$.
\end{itemize}
By taking $z' \in \Gamma z$ that minimises the distance to $w$ (which exists since $\Gamma$ acts properly discontinuously), so we get $z' \in \bar{D}$.

We have that $\quot{\bar{D}}{\Gamma}$ is a punctured torus. In particular, this has no boundary. We later construct a different fundamental domain which has a boundary.
\end{itemize}
\end{example}

\begin{proposition}
Let $\ell$ be a geodesic line and let $x \in \mbb{H}$ such that $x \notin \ell$. There exists a unique $y \in \ell$ such that $d_{\mbb{H}}\prs{x,y} = d_{\mbb{H}}\prs{x,\ell}$, and the geodesic segment $\brs{x,y}$ is orthogonal to $\ell$.
\end{proposition}

\begin{proof}
Assume $z \in \ell$ doesn't satisfy $\brs{x,z} \perp \ell$ and let $y$ be such that $\brs{x,y} \perp \ell$.  By the Pythagorean theorem
\[\cosh\prs{d_{\mbb{H}}\prs{x,z}} = \cosh\prs{d_{\mbb{H}}\prs{x,y}} \cosh\prs{d_{\mbb{H}}}\prs{y,z}\]
where $\cosh\prs{d_{\mbb{H}}}\prs{y,z} > 1$.
\end{proof}

\begin{proposition}
Let $\ell, \ell'$ be ultra-prallel geodeisc lines. There exists a unique $p \in \ell$ and $p' \in \ell'$ such that  $d_{\mbb{H}}\prs{p,p'} = d_{\mbb{H}}\prs{\ell, \ell'}$.
\end{proposition}

\begin{proof}
Existence follows from the fact that the hyperbolic distance increases near infinity, so we can restrict to compact subsets.

For uniqueness, assume there are points $q \in \ell$ and $q' \in \ell'$ not on the orthogonal line to $\mbb{R}$ through the centre of the circle through $\ell$.
Let $y$ be the centre of the circle in which the geodesic $s$ through $q,q'$ passes through, and let $r$ be its Euclidean radius. Then $y^2 = r^2 + t^2 = r^2 + s^2$ and $s > t - x$ where $t$ is the radius of the circle bounded by $\ell$.
\end{proof}

\begin{proposition}
Let $g \in \mrm{PSL}_2\prs{\mbb{R}}$ be hyperbolic and let $\ell,\ell'$ be ultra-parallel geodesics such that $g\prs{\ell} = \ell$, such that $g\prs{z} = kz$ for $k > 1$ and that
\begin{align*}
\ell &= \set{z \in \mbb{H}}{\abs{z} = r} \\
\ell' &= \set{z \in \mbb{H}}{\abs{z} = kr} \text{.}
\end{align*}
Let $p,p'$ be the points minimising the distance between $\ell,\ell'$ and let $w$ be the mid-point of $\brs{p,p'}$, then $w = ir \sqrt{k}$.
Let
\begin{align*}
A &\ceq \set{z \in \mbb{H}}{\abs{z} < r} \\
B &\ceq \set{z \in \mbb{H}}{\abs{z} \in \prs{r, kr}} \\
C &\ceq \set{z \in \mbb{H}}{\abs{z} > kr} \text{.}
\end{align*}
Then for every $z \in A$ it holds that
\[d_{\mbb{H}}\prs{z,w} > d\prs{gz, w} \text{.}\]
\end{proposition}

\begin{proof}
First consider a few cases.
\begin{itemize}
\item
If $z = iy$ and $ky \leq r \sqrt{k}$, the claim is clear.
\item
If $z = iy$ and $ky > r \sqrt{k}$, then \[d_{\mbb{H}}\prs{z,w} = \ln\prs{\frac{r\sqrt{k}}{y}} > \ln\sqrt{k} > \ln\prs{\frac{ky}{r \sqrt{k}}} = \rho\prs{gz, w} \text{.}\]
Let $\gamma = \brs{z,z'}$, then $g\gamma = \brs{gz, gz'}$ because $g$ sends geodesics to geodesics.
Then by the Pythagorean theorem
\begin{align*}
\cosh d_{\mbb{H}} \prs{z,w} &= \cosh d_{\mbb{H}}\prs{z,z'} \cosh d_{\mbb{H}}\prs{z',w}
\\&>
\cosh d_{\mbb{H}} \prs{gz, gz'} \cdot \cosh d_{\mbb{H}} \prs{gz', w}
\\&= \cosh d_{\mbb{H}}\prs{gz, w} \text{.}
\end{align*}
\end{itemize}
\end{proof}

\begin{lemma}[Ping-Pong Lemma]
Let $G$ be a group which acts on a set $X$. Let $A+, A^-, B^+, B^-$ be disjoint subsets of $X$ and let $g,h \in G$ such that the following hold.
\begin{enumerate}
\item $C \ceq X \setminus A^+ \cup A^- \cup B^+ \cup B^- \neq \ns$.
\item For all $n \in \mbb{N}_+$ it holds that
\begin{align*}
g^n\prs{X \setminus A^-} &\subseteq A^+ \\
g^{-n}\prs{X \setminus A^+} &\subseteq A^- \text{.}
\end{align*}
\item For all $n \in \mbb{N}_+$ it holds that
\begin{align*}
h^n\prs{X \setminus B^-} &\subseteq B^+ \\
h^{-n}\prs{X \setminus B^+} &\subseteq B^- \text{.}
\end{align*}
\end{enumerate}
Then $\trs{g,h}$ is a non-abelian free group.
Moreover, if $f \in \trs{g,h}$ is a non-identity element and $c \in C$ then
\begin{enumerate}
\item If $f$ ends with $g$ as a reduced word then $fc \in A^+$.
\item If $f$ ends with $g^{-1}$ as a reduced word then $fc \in A^-$.
\item If $f$ ends with $h$ as a reduced word then $fc \in B^+$.
\item If $f$ ends with $h^{-1}$ as a reduced word then $fc \in B^-$.
\end{enumerate}
\end{lemma}

\begin{proof}
Let $w\prs{x,y}$ be a reduced word different than $1$. We will prove by induction on the length of $w$ that $w\prs{g,h} = f$ satisfies the properties.

In the base case $w \in \set{x, x^{-1}, y, y^{-1}}$ and the claim follows from the assumption.

Assume now that $w\prs{x,y} = z w'$ where $z \in \set{x, x^{-1}, y, y^{-1}}$. Assume WLOG that $z = x$. Since $w$ is reduced, $w'$ does not end with $x^{-1}$. By induction we have $w'\prs{g,h} c \in A^+ \cup B^+ \cup B^{-1}$, which is disjoint from $A^{-1}$, so by assumption $w\prs{g,h} c = g\prs{w'\prs{g,h} c} \in A^+$.
\end{proof}

\begin{example}
Let $g,h \in \mrm{PSL}_2\prs{\mbb{Z}}$, let $g\prs{z} = 2z$ and let $h\prs{z} = \frac{3z + 4}{2z + 3}$.
Consider figure %TODO add ref 1.7
%TODO add fig 1.7
with $A^-, A^+, B^-, B^+$ closed.
$g,h$ satisfy the assumptions of the previous lemma, hence $\trs{g,h}$ is a free group and every orbit intersects $C$ in at most one point.

We claim $C$ is in fact a fundamental domain. Let $p \in \ell$, $p' \in \ell'$, $q \in m$ and $q' \in m'$ such that $d_{\mbb{H}}\prs{\ell, \ell'} = d_{\mbb{H}}\prs{p,p'}$ and $d_{\mbb{H}}\prs{m,m'} = d_{\mbb{H}}\prs{q,q'}$.
Denote $w = i \sqrt{2}$. Then $w$ is the mid-point of both $\brs{p,p'}, \brs{q,q'}$.

By the lemma if $x \in \bar{C}$ then $x$ is in the interior of $A \cup A^- \cup B \cup B^-$.
Hence there exists $f \in \set{g, g^{-1}, h, h^{-1}}$ such that $d\prs{x,m} > d\prs{fx, m}$.
Since the action is properly discontinuous, there exists $y \in G x$ such that $d\prs{m, y} = d\prs{Gx, m}$ so $y$ must belong to $\bar{C}$.

Let $t \in \ell$. Then $t, 2t$ are two points on the boundary of $C$ which are in the same $G$-orbit. We claim that $G \cdot t \cap \bar{C} = \set{t, 2t}$.
Otherwise, there's $f \in G \setminus \set{g}$ such that $f\prs{\bar{C}} \ni 2t$ so $f C \cap C \neq \ns$, a contradiction.
\end{example}

\begin{definition}[Locally Finite Fundamental Domain]
Let $\Gamma \leq \mrm{PSL}_2\prs{\mbb{R}}$ be a Fuchsian group. A fundamental domain $D$ for $\Gamma$ is called locally-finite if every compact subset of $\mbb{H}$ meets only finitely-many of $\set{g \bar{D}}{g \in \Gamma}$.
\end{definition}

\begin{lemma}
$D$ is locally finite if and only if for every sequence $\prs{z_n}_{n \in \mbb{N}}$ of elements of $D$ and every sequence $\prs{g_n}_{n \in \mbb{N}}$ of distinct elements of $\Gamma$, the sequence $\prs{g_n z_n}_{n \in \mbb{N}}$ does not converge.
\end{lemma}

\begin{proof}
Assume there are sequences $\prs{z_n}_{n\ in \mbb{N}}$ and $\prs{g_n}_{n \in \mbb{N}}$ as above such that $\lim_{n\to\infty} g_n z_N = w$. Then any compact neighbourhood of $w$ meets infinitely many $g_n D$ so $D$ isn't locally finite.

Assume now that $D$ is not locally finite. Then there exist a compact subset $K \subseteq D$ and a sequence $\prs{g_n}_{n \in \mbb{N}}$ of distinct elements of $\Gamma$ such that for every $n$, $g_n \bar{D} \cap K \neq \ns$. For every $n \in \mbb{N}$ pick $z_n \in \bar{D} \cap g_n K$, so $g_n z_n \in K$. By passing to a subsequence we can assume that $\prs{g_n z_n}_{n \in \mbb{N}}$ converges to a point $w \in K$. Since every $z_n$ belongs to $\bar{D}$ we can replace each $z_n$ with $z_n' \in D$ such that $\lim_{n \to \infty} g_n z_n = w$.
\end{proof}

\begin{theorem}
Let $\Gamma$ be a Fuchsian group and let $D$ be a fundamental domain for $\Gamma$.
Consider the following diagram.
\[
\begin{tikzcd}
\bar{D} \arrow[r, "\tau"] \arrow[d, "\tilde{\pi}", swap] & \mbb{H} \arrow[d, "\pi"] \\
\quot{\bar{D}}{\Gamma} \arrow[r, "\theta"] & \quot{\mbb{H}}{\Gamma}
\end{tikzcd}
\]
Here $\theta$ is a homeomorphism iff $D$ is locally finite.
\end{theorem}

\begin{proof}
We know that $\theta$ is bijective and continuous. To show $\theta$ is a homeomorphism it is therefore enough to show that if $A \subseteq \quot{\bar{D}}{\Gamma}$ is closed then $\theta\prs{A}$ is closed.
Since $\quot{\mbb{H}}{\Gamma}$ is endowed with the quotient topology, a set $C \subseteq \quot{\mbb{H}}{\Gamma}$ is closed if and only if its inverse image is closed. In our case, this is true if and only if for every sequence $\prs{g_n}_{n \in \mbb{N}} \subseteq \Gamma$ and for every $\prs{z_n}_{n \in \mbb{N}} \subseteq \pi^{-1}\prs{C}$, if $\prs{g_n z_n}_{n \in \mbb{N}}$ converge then $\lim_{n\to\infty} g_n z_n \in \pi^{-1}\prs{c}$. By the lemma, this is equivalent to $D$ being locally-finite.

Let $A \subseteq \quot{\bar{D}}{\Gamma}$ and denote $B = \tilde{\pi}^{-1}\prs{A}$ and $C = \theta\prs{A}$. Then $C$ is closed if and only if for every sequence $\prs{g_n}_{n \in \mbb{N}}$ of elements of $\Gamma$ and every sequence $\prs{z_n}_{n \in \mbb{N}}$ of elements of $B$ such that $\prs{g_n z_n}_{n \in \mbb{N}}$ converges to an element $w \in B = \pi^{-1}\prs{C} \cap \bar{D}$.

Assume that $D$ is locally finite and $A \leq \quot{\bar{D}}{\Gamma}$ is closed. Then $B = \tilde{\pi}^{-1}\prs{A}$ is closed in $\bar{D}$ and thus in $\mbb{H}$. Let $\prs{z_n}_{n \in \mbb{N}}$ be a sequence of elements of $B$ and $\prs{g_n}_{n \in \mbb{N}}$ a sequence of elements of $\Gamma$ such that $\lim_{n\to\infty} g_n z_n = w \in \bar{D}$. Since $D$ is locally finite, by passing to a subsequence we may assume that there's $g \in \Gamma$ such that $g_n = g$ for every $n \in \mbb{N}$. Then $\lim_{n\to\infty} g z_n = w$ so $\lim_{n\to\infty} z_n = g^{-1} w \in B$ where this belongs to $B$ since $B$ is closed.
Because $B$ is invariant under $\Gamma$ we get $w = g\prs{g^{-1}w} \in B$.

Assume now that $D$ is not locally finite. Let $\prs{g_n}_{n \in \mbb{N}}$ be a sequence of distinct elements of $\Gamma$ and $\prs{z_n}_{n \in \mbb{n}}$ a sequence of elements  of $D$ such that $\lim_{n\to\infty} g_n z_N = w \in \bar{D}$. Since the action is properly discontinuous. By passing to a subsequence we may assume that $\prs{z_n}_{n \in \mbb{N}}$ are distinct and that $w \notin \prs{z_n}_{n \in \mbb{N}}$.
We claim that the set $B \ceq \set{z_n}{n \in \mbb{N}}$ is closed. Otherwise, by passing to a subsequence we may assume that $\lim_{n\to\infty} z_n = z$.
Then $\lim_{n\to\infty} g_n z_n = w$ so $\lim_{n\to\infty} g_n^{-1} w = z$, which is impossible since the $\Gamma$-action is properly-discontinuous. Denote $A = \tilde{\pi}\prs{B}$. Since $B \subseteq D$, we have $\tilde{\pi}^{-1}\prs{A} = B$, so $A$ is closed. But, $w \in \bar{D} \setminus B$ so $w \notin \pi^{-1}\prs{\theta\prs{A}}$, so $\theta\prs{A}$ is not closed.
\end{proof}

\end{document}