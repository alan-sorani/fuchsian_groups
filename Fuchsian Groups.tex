\documentclass[10pt, twoside]{book}

%%%%%%%%%
% Maths %
%%%%%%%%%

\usepackage{math-fonts}
\usepackage{math-graphics}
\usepackage{math-symbols}
\usepackage{math-theorems}

%%%%%%%%%
% Title %
%%%%%%%%%

\title{Lecture Notes to Fuchsian Groups \\ \large{Winter 2020, Technion}}
\author{Lectures by Chen Meiri \\ \small{Typed by Elad Tzorani}}
\date{\today}

\begin{document}

\maketitle
\tableofcontents

\chapter{Preliminaries}

\section{The Hyperbolic Plane}

\subsection{The Riemann Sphere}

\begin{definition}[The Riemann Sphere]
The \emph{Riemann sphere} is a one-dimensional complex manifold, denoted $\hat{\mbb{C}} = \mbb{C} \cup \set{\infty}$, the charts of which are the following.
\begin{align*}
U_1 &= \prs{\mbb{C}, f_1} \\
U_2 &= \prs{\hat{\mbb{C}} \setminus \set{0}, f_2}
\end{align*}
where
\begin{align*}
f_1 \colon \mbb{C} &\to \mbb{C} \\
z &\mapsto z
\end{align*}
and
\begin{align*}
f_2 \colon \mbb{C} &\to \hat{\mbb{C}} \setminus \set{0} \\
z &\mapsto \frac{1}{z} \text{.}
\end{align*}
\end{definition}

\begin{definition}[Möbius Transformation]
A map $T \colon \hat{\mbb{C}} \to \hat{\mbb{C}}$ of the form
\[z \mapsto \frac{az + b}{cz + d}\]
where $ad - bc \neq 0$ is called a \emph{Möbius transformation}.
\end{definition}

\begin{notation}
\begin{enumerate}
\item We denote the image of $\pmat{a & b \\ c & d} \in \mrm{GL}_2\prs{\mbb{C}}$ in $\mrm{PGL}_2\prs{\mbb{C}}$ by $\bmat{a & b \\ c & d}$.
\item For every $g = \bmat{a & b \\ c & d} \in \mrm{PGL}_2\prs{\mbb{C}}$, we denote by $T_g$ the Möbius transformation $z \mapsto \frac{az + b}{cz + d}$.
\end{enumerate}
\end{notation}

\begin{lemma}
The set of Möbius transformations is a group under composition, and the map $g \mapsto T_g$ is an isomorphism between $\mrm{PGL}_2\prs{\mbb{C}}$ and the group of Möbius transformation.
\end{lemma}

\begin{proof}
It holds that
\begin{align*}
T_{g_1} \circ T_{g_2} \prs{z} &=  \frac{a_1 \prs{\frac{a_2 z + b_2}{c_2 z + d_2}} + 1}{c_1 \prs{\frac{a_2 z + b_2}{c_2 z + d_2}} + d_1}
\\&= \frac{\prs{a_1 a_2 + b_1 c_2} z + \prs{a_1 b_2 + b_1 d_2}}{\prs{c_1 a_2 + d_1 c_2} z + \prs{c_1 b_2 + d_1 d_2}}
\\&= T_{g_1 g_2}\prs{z} \text{.}
\end{align*}
In particular, $T_{g^{-1}}$ is the inverse of $T_g$.

The rest of the proof is clear.
\end{proof}

\begin{definition}[Generalised Circle]
A generalised circle in $\mbb{C}$ is either an Euclidean circle or an Euclidean straight line.
\end{definition}

\begin{lemma}
Let $T \colon \hat{\mbb{C}} \to \hat{\mbb{C}}$ be a Möbius transformation. Then
\begin{enumerate}
\item $T$ is an endomorphism of $\hat{\mbb{C}}$.
\item $T$ is conformal.
\item $T$ sends generalised circles to generalised circles.
\end{enumerate}
\end{lemma}

\subsection{Models of the Hyperbolic Plane}

\begin{definition}[The Upper Half Plane Model for the Hyperbolic Plane]
\begin{enumerate}
\item As a set, define $\mbb{H} \ceq \set{z \in \mbb{C}}{\Im\prs{z} > 0}$.
\item Let $\gamma \colon \brs{0,1} \to \mbb{H}$ be a piecewise continuously differentiable path given by $\gamma\prs{t} = x\prs{t} + iy\prs{t}$ for real functions $x\prs{t}, y\prs{t}$.
The \emph{hyperbolic length} of $\gamma$ is given by
\[h\prs{\gamma} \ceq \int_0^1 \frac{\sqrt{\prs{\frac{\diff x}{\diff t}}^2 + \prs{\frac{\diff y}{\diff t}}^2}}{y\prs{t}} \diff t = \int_0^1 \frac{\abs{\frac{\diff \gamma}{\diff t}}}{y\prs{t}} \diff t \text{.}\]
\item The \emph{hyperbolic distance} $\rho\prs{z,w}$ between two points $z,w \in \mbb{H}$ is defined as $\inf_\gamma h\prs{\gamma}$ where the infimum is taken over all piecewise continuously differentiable paths $\gamma$ from $z$ to $w$.
\end{enumerate}
\end{definition}

\begin{remark}
$\mbb{H}$ is a Riemann surface where for every $z \in \mbb{H}$, the inner product of $T_z H$ is given by
\[\prs{\prs{x_1, y_1},\prs{x_2,y_2}} = \frac{x_1 x_2 + y_1 y_2}{\prs{\Im z}^2} \text{.}\]
In particular, Euclidean angles are equal to hyperbolic angles.
\end{remark}

\begin{definition}[The Disc Model for the Hyperbolic Plane]
\begin{enumerate}
\item As a set, define $\mbb{U} \ceq \set{z \in \mbb{C}}{\abs{z} < 1}$.
\item Let $\gamma \colon \brs{0,1} \to \mbb{U}$ be a piecewise continuously differentiable path. The \emph{hyperbolic length} of $\gamma$ is given by
\[h_u\prs{\gamma} \ceq \int_0^1 \frac{2 \abs{\frac{\diff \gamma}{\diff t}}}{1 - \abs{\gamma\prs{t}}^2} \diff t \text{.}\]

\item The \emph{hyperbolic distance} $\rho_u\prs{z,w}$ between $z,w \in \mbb{U}$ is defined to be $\inf_\gamma h\prs{\gamma}$ where the infimum is taken over all piecewise continuously differentiable paths from $z$ to $w$.
\end{enumerate}
\end{definition}

\begin{remark}
It is clear that hyperbolic circles around $0$ are exactly Euclidean circles around it (with a generally different radius).
\end{remark}

\begin{remark}
Rotations around $0$ are isometries in the disc model.
\end{remark}

\begin{lemma}
Let $\pi$ be the Möbius transformation defined by
\[\pi\prs{z} = \frac{iz + 1}{z + i} \text{.}\]
Then
\begin{enumerate}
\item $\pi$ is a bijection from $\mbb{H}$ to $\mbb{U}$.
\item For every piecewise continuously differentiable path $\gamma \colon \brs{0,1} \to \mbb{H}$,it holds that $h_u\prs{\pi\prs{\gamma}} = h\prs{\gamma}$.
In particular, $\pi$ is an isometry.
\end{enumerate}
\end{lemma}

\begin{proof}
\begin{enumerate}
\item It holds that
\begin{align*}
\pi\prs{-1} &= -1 \\
\pi\prs{0} &= -i \\
\pi\prs{1} = 1 \text{.}
\end{align*}
Since Möbius transformations send generalised circles to generalised circles we get that $\pi$ sends $\mbb{R}$ to the unit circle.
Since $\pi\prs{i} = 0$ and $\pi$ is a homeomorphism of the Riemann sphere, we get the result.

\item Let $\gamma \colon \brs{0,1} \to \mbb{H}$ be a piecewise continuously differentiable path. Denote $\psi = \pi^{-1}$ and $\delta = \pi\prs{\gamma}$.
Then
\[\psi\prs{z} = \frac{iz - 1}{-z + i} = \frac{\prs{iz - 1}{- \bar{z} - i}}{\prs{-z + i}{-\bar{z} - i}} = \frac{\prs{z + \bar{z}} + i \prs{1 - \abs{z}^2}}{\abs{-z + i}^2} \text{.}\]
So,
\[\Im\prs{\psi\prs{z}} = \frac{1 - \abs{z}^2}{\abs{-z + i}^2} \text{.}\]
Since
\[\frac{\diff \psi}{\diff z} = \frac{-2}{\prs{-z+i}^2} \text{,}\]
we get that
\begin{align*}
h\prs{\gamma} &= \int_0^1 \frac{\abs{\frac{\diff \gamma}{\diff t}}}{\Im\prs{\gamma\prs{t}}} \diff t
\\&=
\int_0^1 \frac{\abs{\frac{\diff \psi\prs{\delta}}{\diff t}}}{\Im \prs{\psi\prs{\delta\prs{t}}}} \diff t
\\&=
\int_0^1 \frac{\abs{\frac{\diff \psi}{\diff z} \prs{\delta\prs{t}} \frac{\diff \delta}{\diff t}}}{\Im \prs{\psi\prs{\delta\prs{t}}}}
\\&= \int_0^1 \frac{2 \abs{\frac{\diff \delta}{\diff t}}}{1 - \abs{\delta\prs{t}}^2} \diff t
\\&= h_u\prs{\delta} \text{.}
\end{align*}

\end{enumerate}
\end{proof}

\subsection{Isometries of the Hyperbolic Plane}

\begin{lemma}
For every $g \in \bmat{a & b \\ c & d} \in \mrm{PSL}_2\prs{\mbb{R}}$ it holds that
\[T_g\prs{\mbb{H}} \subseteq \mbb{H} \text{.}\]
\end{lemma}

\begin{proof}
It's enough to show the inclusion $T_g\prs{\mbb{H}} \subseteq \mbb{H}$ since then
\[T_{g^{-1}}\prs{\mbb{H}} = \prs{T_g}^{-1} \prs{\mbb{H}} \subseteq \mbb{H}\]
which implies $T_g\prs{\mbb{H}} \supseteq \mbb{H}$ by applying $T_g$.

Now, we have
\begin{align*}
T_g\prs{z} &= \frac{az + b}{cz + d}
\\&=
\frac{\prs{az + b}\prs{c \bar{z} + d}}{\abs{cz + d}^2}
\\&=
\frac{ac \abs{z}^2 + adz + bc \bar{z} + bg}{\abs{cz + d}^2} \text{.}
\end{align*}
Thus,
\begin{align*}
\Im\prs{T_g\prs{z}} &= \frac{T_g\prs{z} - \overline{T_g\prs{z}}}{2 i}
\\&=
\frac{\prs{ad - bc} z - \prs{ad - bc} \bar{z}}{2i \abs{cz + d}^2}
\\&\underset{ad - bc = 1}{=}
\frac{\Im\prs{z}}{\abs{cz + d}^2} \text{.}
\end{align*}

\end{proof}

This lemma allows us to identify $\mrm{PSL}_2\prs{\mbb{R}}$ as a subgroup of $\mrm{Sym}\prs{\mbb{H}}$. The next lemma shows that even more is true.

\begin{lemma}
$\mrm{PSL}_2\prs{\mbb{R}} \subseteq \mrm{Isom}\prs{\mbb{H}}$.
\end{lemma}

\begin{proof}
It's enough to show that for every $g \in \mrm{PSL}_2\prs{\mbb{R}}$ and every piecewise continuously differentiable path $\gamma$ it holds that $h\prs{\gamma} = h\prs{T_g\prs{\gamma}}$.
Denote $T = T_g$ and $\delta = T\prs{\gamma}$. Then
\begin{align*}
h\prs{\delta} &= \int_0^1 \frac{\abs{\frac{\diff \delta}{\diff t}}}{\Im\prs{\delta\prs{t}}} \diff t
\\&=
\int_0^1 \frac{\abs{\frac{\diff T}{\diff z} \prs{\gamma\prs{t}} \frac{\diff \gamma}{\diff t}}}{\Im\prs{\delta\prs{t}}} \diff t
\\&\underset{\star}{=}
\int_0^1 \frac{\abs{\frac{\diff \gamma}{\diff t}}}{\gamma\prs{t}} \diff t
\\&=
h\prs{\gamma}
\end{align*}
where $\star$ follows from
\begin{align*}
\Im\prs{T_g\prs{z}} &= \frac{\Im \prs{z}}{\abs{cz + d}^2} \oplus \frac{\diff T}{\diff z} \\&=
\frac{a\prs{cz + d} - c\prs{az + b}}{\prs{cz + d}^2}
\\&= \frac{1}{\prs{cz + d}^2} \text{.}
\end{align*}
\end{proof}

\begin{corollary}
$\mrm{Isom}\prs{\mbb{H}}$ acts transitively on $\mbb{H}$.
\end{corollary}

\begin{proof}
It's enough to show that for every $z \in \mbb{H}$ there's $g \in \mrm{PSL}_2\prs{\mbb{R}}$ such that $T_g\prs{z} = i$.

If $z = x + yi$, take $g = \pmat{\frac{1}{\sqrt{y}} & - \frac{x}{\sqrt{y}} \\ 0 & \frac{1}{\sqrt{y}}}$, then
\[T_g\prs{z} = \frac{1}{y} \prs{x + yi} - \frac{x}{y} = i \text{.}\]
\end{proof}

\begin{lemma}
Let $\pi \colon \mbb{H} \to \mbb{U}$ be the isometry $z \mapsto \frac{iz + 1}{z + i}$ which we defined previously.
Then
\[\set{\pi T_g \pi^{-1}}{g \in \mrm{PSL}_2\prs{\mbb{R}}} = \set{\pmat{r & s \\ \bar{r} & \bar{s}}}{\substack{r,s \in \mbb{C} \\ \abs{r}^2 - \abs{s}^2 = 1}} \text{.}\]
In particular, by taking $r = e^{i \theta}$ and $s = 0$ we see that the action of $\mrm{PSL}_2\prs{\mbb{R}}$ on $\mbb{U}$ contains all the rotations around $0$.
\end{lemma}

\begin{proof}
It holds that
\[\pmat{i & 1 \\ 1 & i} \pmat{a & b \\ c & d} \pmat{i & -1 \\ -1 & i} = \frac{1}{2} \pmat{\prs{a+d} + i\prs{b-c} & b + c + i\prs{a-d} \\ \prs{b+c} - i \prs{a+d} & \prs{a+d} - i\prs{b-c}} \text{.}\]
Now, $\prs{a+d, a-d, b+c, b-c}$ can be any $4$-tuple. Specifically, for every $r,s \in \mbb{C}$ we have $a,b,c,d \in \mbb{R}$ such that $\pi T_g \pi^{-1} = \pmat{r & s \\ \bar{s} & \bar{r}}$, and by the equality from the determinants we get that $\bmat{a & b \\ c & d } \in \mrm{PSL}_2\prs{\mbb{R}}$.
\end{proof}

\begin{corollary}
Let $z_1, z_2, w_1, w_2 \in \mbb{H}$ be such that $\rho\prs{z_1, w_1} = \rho\prs{z_2, w_2}$, then there exists $g \in \mrm{PSL}_2\prs{\mbb{R}}$ such that $T_g\prs{z_1} = z_2$ and $T_g\prs{w_1} = w_2$.
\end{corollary}

\begin{proof}
Since $\mrm{PSL}_2\prs{\mbb{R}}$ acts transitively on $\mbb{H}$ we can assume that $z_1 = z_2$ and show that $\mrm{Stab}\prs{z_1}$ acts transitively on $\set{w \in \mbb{H}}{\rho\prs{z_1, w} = \rho\prs{z_1, w_1}}$.
We already showed this in the disc model, in the case $z_1 = i$.
\end{proof}

\begin{definition}
Let $\prs{X,d}$ be a metric space.
\begin{enumerate}
\item Let $x,y \in X$. A path $\gamma \colon \brs{a,b} \to X$ which joins $x$ and $y$ is called a \emph{geodesic segment} if for every $a \leq t_1 \leq t_2 \leq b$ it holds that $\abs{t_2 - t_1} = d\prs{\gamma\prs{t_1}, \gamma\prs{t_2}}$.

\item A path $\gamma \colon \mbb{R} \to X$ is called a \emph{geodesic line} if for every $a < b$ it holds that $\left. \gamma \right|_{\brs{a,b}}$ is a geodesic segment.
\end{enumerate}
\end{definition}

\begin{remark}
Let $\gamma$ be a geodesic segment or line.
Then $\gamma$ is determined by the image of $\gamma$ up to a composition with an isometry of $\mrm{R}$. Thus, we can identify geodesic segments and lines with their image up to orientation.
\end{remark}

\begin{lemma}
Let $b > a > 0$ be real numbers. Then $\set{iy}{a \leq y \leq b}$ is the unique geodesic segment between $ia$ and $ib$ and $\set{iy}{y > 0}$ is the unique geodesic line through $ia$ and $ib$.
\end{lemma}

\begin{proof}
We begin with the first part of the lemma.
Let $\gamma \colon \brs{0,1} \to \mbb{H}$ be a piecewise continuously differentiable path joining $ia$ and $ib$. For $t \in \brs{0,1}$ denote $\gamma\prs{t} = x\prs{t} + iy\prs{t}$ where $x\prs{t},y\prs{t} \in \mbb{R}$.
Then
\begin{align*}
h\prs{\gamma} &= \int_0^1 \frac{\sqrt{\prs{\frac{\diff x}{\diff t}}^2 + \prs{\frac{\diff y}{\diff t}}^2}}{y\prs{t}} \diff t
\\&\underset{\star}{\geq}
\int_0^1 \frac{\abs{\frac{\diff y}{\diff t}}}{y\prs{t}} \diff t
\\&\geq
\int_0^1 \frac{\frac{\diff y}{\diff t}}{y\prs{t}} \diff t
\\&=
\ln\prs{\frac{b}{a}} \text{.}
\end{align*}
Thus, $\rho\prs{ia, ib} \geq \ln\prs{\frac{b}{a}}$. If $y\prs{t} = i\prs{\prs{b-a}t + a}$, the above inequalities are equalities so $\rho\prs{ia, ib} = \ln\prs{\frac{b}{a}}$. The inequality $\star$ is an equality if and only if $x\prs{t} = 0$ for all $t \in \brs{a,b}$. It follows that the unique geodesic segment between $a$ and $b$ is $\set{iy}{a \leq y \leq b}$.

Now, it is clear that $\set{iy}{y > 0}$ is a geodesic line which passes through $ia$ and $ib$.
We want to show it's unique.

Assume towards a contradiction that there exists a geodesic line $\ell$ between $ia$ and $ib$ which isn't the positive part of the $y$-axis.
Then there's $z = x + iy \in \ell$ for which $x \neq 0$ and $\rho\prs{z, ia} > \rho\prs{z, ib}$. By the previous lemma, there exists $g \in \mrm{PSL}_2\prs{\mbb{R}}$ such that $T_g\prs{ia} = ia$ and $T_g\prs{z} \in i \mbb{R}$. Since $T_g$ sends generalised circles to generalised circles, $T_g\prs{ib} \notin i \mbb{R}$. Indeed, otherwise the image of the segment between $ia$ and $ib$ would belong to $i \mbb{R}$, and since $T_g$ sends generalised circles to generalised circles, it would send $i\mbb{R}$ to itself.

We get that there exists a geodesic between $ia$ and $T_g\prs{z} = ic$ which is not contained in $i\mbb{R}$, and this is impossible.
\end{proof}

\begin{theorem}

\begin{enumerate}
\item
Every distinct points $z,w \in \mbb{H}$ are contained in a unique geodesic segment and a unique geodesic line.

\item The geodesics in $\mbb{H}$ are semicircles and lines orthogonal to the real axis.
\end{enumerate}
\end{theorem}

\begin{proof}
\begin{enumerate}
\item
For every $g \in \mrm{PSL}_2\prs{\mbb{R}}$ it holds that $T_g\prs{\mbb{R} \cup \set{\infty}} = \mbb{R} \cup \set{\infty}$. If $z,w \in \mbb{H}$, by a previous lemma there exists $g \in \mrm{PSL}_2\prs{\mbb{R}}$ such that $T_g\prs{z} = ia$ and $T_g\prs{w} = ib$ for some $a,b \in \mbb{R}_+$.
Thus, $T_g^{-1} \prs{\brs{ia, ib}}$ is the unique geodesic segment between $z$ and $w$.

\item This follows from the fact that Möbius circles are conformal, send generalised circles to generalised circles, and sends $\mbb{R} \cup \set{\infty}$ to itself.      
\end{enumerate}
\end{proof}

\begin{corollary}
The geodesic segment in $\mbb{U}$ are segments of straight lines through zero or arcs of circles which are orthogonal to the unit circles.
\end{corollary}

\begin{theorem}
Let $z,w \in \mbb{H}$. Then
\begin{align*}
\sinh\prs{\frac{1}{2} \rho\prs{z,w}} = \frac{\abs{z-w}}{2 \prs {\Im\prs{z} \Im\prs{w}}^{\frac{1}{2}}} \text{.}
\end{align*}
\end{theorem}

\begin{proof}
Since $\mrm{PSL}_2\prs{\mbb{R}} \subseteq \mrm{Isom}\prs{\mbb{H}}$, the left side of the equation is invariant under the action of $\mrm{PSL}_2\prs{\mbb{R}}$. We first show that the right side is also invariant.

It's clear that the right side is invariant under maps of the form $z \mapsto az + b$ for $a,b \in \mbb{R}$. Since $\mrm{PSL}_2\prs{\mbb{R}}$ (viewed as a group of Möbius transformations) is generated by maps of the forms
\begin{align*}
z &\mapsto az + b, \, a,b\in\mbb{R} \\
z &\mapsto - \frac{1}{z}
\end{align*}
it's enough to show that the right side is invariant under these maps.

The right side is indeed invariant under $\frac{1}{z}$ since
\begin{align*}
\frac{\abs{\frac{1}{z} - \frac{1}{w}}}{2 \prs{\Im\prs{\frac{1}{z}} \Im\prs{\frac{1}{w}}}^{\frac{1}{2}}}
&=
\frac{\abs{\frac{z-w}{zw}}}{2 \prs{\Im\prs{\frac{z}{\abs{z}^2}} \Im\prs{\frac{w}{\abs{w}^2}}}^{\frac{1}{2}}}
\\&=
\frac{\abs{z-w}}{2 \prs{\Im\prs{z}\Im\prs{w}}^{\frac{1}{2}}} \text{.}
\end{align*}

Since both sides of the equation are invariant under the action of $\mrm{PSL}_2\prs{\mbb{R}}$, it's enough to prove the equality for $z=i$ and $w = ir$ for some $r \in \mbb{R}_+$.
Indeed,
\begin{align*}
\sinh\prs{\frac{1}{2} \rho\prs{i, ir}} &=
\sinh\prs{\frac{1}{2} \abs{\ln r}}
\\&=
\frac{\abs{\sqrt{r} - \frac{1}{\sqrt{r}}}}{2}
\\&=
\frac{\abs{r-1}}{2 \sqrt{r}}
\\&=
\frac{\abs{i - ir}}{2 \prs{\Im\prs{i} \Im\prs{ir}}^{\frac{1}{2}}} \text{.}
\end{align*}
\end{proof}

\begin{corollary}
\begin{enumerate}
\item The hyperbolic topology is equal to the Euclidean topology.
\item $\mbb{H}$ is a complete metric space.
\end{enumerate}
\end{corollary}

\begin{proof}
\begin{enumerate}
\item Let $z \in \mbb{H}$. If $\abs{\Im\prs{z} - \Im\prs{w}} < \frac{1}{2} \Im\prs{z}$ then
\[\frac{\abs{z-w}}{\sqrt{6} \Im\prs{z}} \leq \sinh \prs{\frac{1}{2} \rho\prs{z,w}} \leq \frac{\abs{z-w}}{\sqrt{2} \Im\prs{z}} \text{.}\]
\item We show that $\mbb{U}$ is a complete metric space, which implies the result since there's an isometry between $\mbb{U}$ and $\mbb{H}$.
Let $z,w \in \mbb{U}$, we have
\begin{align*}\label{equation:H_is_complete}
\sinh^2 \prs{\frac{1}{2} \rho\prs{z,w}} = \frac{\abs{z-w}^2}{\prs{1-\abs{z}^2}\prs{z-\abs{w}^2}} \text{.}
\end{align*}
Let $\prs{z_n}_{n \in \mbb{N}}$ be a hyperbolic Cauchy sequence. Then it's bounded in the hyperbolic metric, and \eqref{equation:H_is_complete} implies that it does not have a limit point on the unit circle and so is contained in a compact subset of the unit circle.

The result follows since \eqref{equation:H_is_complete} implies that on such a subset the hyperbolic and Euclidean metric are Lipschitz equivalent. 

\end{enumerate}
\end{proof}

\begin{exercise}
Prove that if $z,w \in \mbb{U}$ then
\[\sinh^2\prs{\frac{1}{2}\rho\prs{z,w}} = \frac{\abs{z-w}^2}{\prs{1 - \abs{z}^2} \prs{1 - \abs{w}^2}} \text{.}\]
\end{exercise}

\begin{theorem}
Let $\tau \colon \mbb{H} \to \mbb{H}$ be $\tau\prs{z} = -\bar{z}$. Then $\mrm{Isom}\prs{\mbb{H}} = \mrm{PSL}_2\prs{\mbb{R}} \rtimes \trs{\tau}$.
In particular, $\mrm{PSL}_2\prs{\mbb{R}}$ is a normal index two subgroup of $\mrm{Isom}\prs{\mbb{H}}$.
\end{theorem}

\begin{proof}
Clearly, $\tau$ is of order two. Since every index two subgroup is normal, it is enough to prove that for every hyperbolic isometry $s \in \mrm{Isom}\prs{\mbb{H}}$ there exists $g \in \mrm{PSL}_2\prs{\mbb{R}}$ such that $sT_g$ is either the identity or $\tau$.

There's $g \in \mrm{PSL}_2\prs{\mbb{R}}$ such that $T_g\prs{i} = s^{-1}\prs{i}$ and $T_g\prs{2i} = s^{-2}\prs{2i}$ (since $\mrm{PSL}_2\prs{\mbb{R}}$ is $2$-transitive). Then
$s T_g\prs{i} = i$ and $sT_g\prs{2i} = 2i$. Since isometries send geodesics to geodesics, for every $t>0$ it holds that $s T_g\prs{t i} = ti$.

Denote
\begin{align*}
U_+ &\ceq \set{z \in \mbb{H}}{\Re\prs{z} > 0} \\
U_{-} &\ceq \set{z \in \mbb{H}}{\Re\prs{z} < 0}\text{.}
\end{align*}
Since $s T_g$ is continuous it follows that $s T_g \prs{U_+} \subseteq U_+$ or $s T_g\prs{U_+} \subseteq U_-$.
In the first case denote $R \ceq s T_g$, and in the second case denote $R \ceq \tau s T_g$. In either case, $R\prs{U_+} \subseteq U_+$.

In order to finish the proof, we want to show that $R = \id$. For every $t > 0 $ we have
\begin{align*}
\frac{\abs{it - w}}{2 \prs{t \Im\prs{w}}^{\frac{1}{2}}} &= \sinh\prs{\frac{1}{2} \rho\prs{it, w}}
\\&=
\sinh\prs{\frac{1}{2} \rho\prs{R\prs{it}, R\prs{w}}}
\\&=
\sinh\prs{\frac{1}{2} \rho\prs{it, R\prs{w}}}
\\&=
\frac{\abs{it - R\prs{w}}}{2 \prs{t \Im\prs{R\prs{w}}}^{\frac{1}{2}}} \text{.}
\end{align*}
So,
\[\abs{it - w}^2 \Im\prs{R\prs{w}} = \abs{it - R\prs{w}}^2 \Im\prs{w} \text{.}\]
This holds for every $t$, which implies together with
\[\Im\prs{R\prs{w}} = \lim_{t \to \infty} \frac{\abs{it - w}^2 \Im\prs{R\prs{w}}}{t^2}\]
that
\[\Im\prs{R\prs{w}} = \lim_{t \to \infty} \frac{\abs{it - R\prs{w}}^2 \cdot \Im\prs{w}}{t^2} = \Im\prs{w} \text{.}\]
Now, for every $t > 0$ we get
\[\abs{it - w} = \abs{it - R\prs{w}}\]
which implies
$w = R\prs{w}$
or
$w = - \overline{R\prs{w}}$.
The latter case is impossible since $R\prs{U_+} \subseteq  U_+$.
\end{proof}

\begin{corollary}
Every element of $\mrm{Isom}\prs{\mbb{H}}$ is either conformal or anti-conformal.

An element of $\mrm{Isom}\prs{\mbb{H}}$ is conformal if and only if it belongs to $\mrm{PSL}_2\prs{\mbb{R}}$.
\end{corollary}

\begin{definition}
Let $\hat{\mbb{C}}$ be the Riemann sphere. The cross ratio of distinct points $z_1, z_2, z_3, z_4 \in \hat{\mbb{C}}$ is
\[\prs{z_1, z_2 : z_3, z_3} \ceq \frac{\prs{z_1 - z_2}{z_3 - z_4}}{\prs{z_2 - z_3}\prs{z_4 - z_1}} \text{.}\]
\end{definition}

\begin{lemma}
Möbius transformations preserve the cross ratio.
\end{lemma}

\begin{proof}
We prove this when $z_1, z_2, z_3, z_4 \in \mbb{C} \setminus \set{0}$. The other cases are left as exercise.

It is clear that maps of the form $z \mapsto az + b$, with $a \neq 0$, preserve the cross-ratio. Thus it's enough to prove that the map $z \mapsto -\frac{1}{z}$ preserves the cross-ratio.
Indeed,
\begin{align*}
\prs{z_1, z_2 ; z_3, z_4} &=
\frac{\prs{z_1 - z_2}{z_3 - z_4}}{\prs{z_2 - z_3}\prs{z_4 - z_1}}
\\&=
\frac{\prs{\frac{z_1 - z_2}{z_1 z_2}}{\frac{z_3 - z_4}{z_3 z_4}}}{\prs{\frac{z_2 - z_3}{z_2 z_3}}\prs{\frac{z_4 - z_1}{z_1 z_4}}}
\\&=
\frac{\prs{\frac{1}{z_1} - \frac{1}{z_2}}\prs{\frac{1}{z_3} - \frac{1}{z_4}}}{\prs{\frac{1}{z_2} - \frac{1}{z_3}}\prs{\frac{1}{z_4} - \frac{1}{z_1}}}
\\&=
\prs{\frac{1}{z_1}, \frac{1}{z_2} ; \frac{1}{z_3}, \frac{1}{z_4}} \text{.}
\end{align*}
\end{proof}

\begin{theorem}
Let $z,w \in \mbb{H}$ and let the geodesic joining $z,w$ have have end points $z^*$ and $w^*$ in $\mbb{R} \cup \set{\infty}$, chosen in a way that $z$ lies between $z^*$ and $w$. Then
\[\rho\prs{z,w} = \ln\prs{\prs{w, z^*; z, w^*}} \text{.}\]
\end{theorem}

\begin{proof}
Since both sides are invariant to the action of $\mrm{PSL}_2\prs{\mbb{R}}$ we can assume that $z = i$ and $w = ri$ with $r > 1$. Then $z^* = 0$ and $w^* = \infty$, so $r = \prs{w, z^*, z, w^*}$ and $\rho\prs{i,ir} = \ln\prs{r}$.
\end{proof}

\section{The Gauss-Bonnet Formula}

\begin{definition}[Hyperbolic Measure]
We define a measure $\mu$ on subsets of $\mbb{H}$ by
\[\mu\prs{A} = \int_A \frac{\diff x \diff y}{y^2}\]
for which this exists.
\end{definition}

\begin{theorem}
The hyperbolic area is invariant under $\mrm{PSL}_2\prs{\mbb{R}}$.
\end{theorem}

\begin{proof}
Let $f \colon \mbb{C} \to \mbb{C}$ given by
\[f\prs{x+iy} = u\prs{x,y} + i v\prs{x,y}\]
where $u,v \colon \mbb{C} \to \mbb{R}$.

By Cauchy-Riemann 
%TODO fill this in!
\begin{align*}
\frac{\del\prs{u,v}}{\del\prs{x,y}} &= \frac{\del u}{\del x} \frac{\del v}{\del y} - \frac{\del u}{\del y} \frac{\del v}{\del x}
\\&=
\prs{\frac{\del u}{\del x}}^2 + \prs{\frac{\del v}{\del y}}^2
\\&= \ldots
\end{align*}

Let $g = \bmat{a & b \\ c & d} \in \mrm{PSL}_2\prs{\mbb{R}}$. Recall that
\begin{align*}
\abs{\frac{\diff T_g}{\diff z}} &= \frac{1}{\abs{cz + d}^2} \\
\Im\prs{T_g\prs{z}} &= \frac{\Im \prs{z}}{\abs{cz + d}^2} \text{.}
\end{align*}
Then
\[T_g\prs{x + iy} = u\prs{x,y} + i v\prs{x,y}\]
so
\begin{align*}
\mu\prs{T_g\prs{A}} &= \int_{T_g\prs{A}} \frac{\diff u \diff v}{v^2}
\\&=
\int_A \frac{\del \prs{u,v}}{\del \prs{x,y}} \frac{\diff x \diff y}{v^2}
\\&=
\int_A \frac{1}{\abs{cz + d}^4 \cdot} \frac{\abs{cz + d}^4}{y \diff x \diff y}
\\&=
\mu\prs{A} \text{.}
\end{align*}
\end{proof}

\begin{definition}[$\tilde{\mbb{H}}$]
\begin{enumerate}
\item Define $\tilde{\mbb{H}} = \mbb{H} \cup \mbb{R} \cup \set{\infty}$.
\item A hyperbolic $n$-sided polygon is a closed subset of $\tilde{\mbb{H}}$ bounded by the closure of $n$ hyperbolic geodesic segments or rays.
\item A side of a polygon is the closure of a geodesic segment or ray which bounds to polygon.
\item A point $z \in \tilde{\mbb{H}}$ is called a vertex if it is the intersection of two distinct sides.
\end{enumerate}
\end{definition}

\begin{example}
There rare four types of hyperbolic triangles, which depend on the number of vertices on the boundary.
%TODO add image
\end{example}

\begin{theorem}[Gauss-Bonnet]
Let $\Delta$ be a hyperbolic triangle with angles $\alpha,\beta,\gamma$. Then
\[\mu\prs{\Gamma} = \pi - \alpha - \beta - \gamma \text{.}\]
\end{theorem}

\begin{proof}
First assume that $\Delta$ has a vertex on the boundary. Since $\mrm{PSL}_2\prs{\mbb{R}}$ preserves area, we may assume that this vertex is $\infty$.
Thus, two sides are given by equations $x = a$ and $x = b$ (and assume $a < b$).
By applying a transformation of the form
\[z \mapsto \lambda z + k\]
where $\lambda > 0$ and $k \in \mbb{R}$, we can assume that the third side of $\Gamma$ is an arc on the geodesic $\abs{z}^2 = 1$.

Pass segments from $0$ to the vertices of the triangle and call the angles between these and the real axis $\alpha$ and $\beta$.
Then
\begin{align*}
\mu\prs{\Delta} &=
\int_\Delta \frac{\diff x \diff y}{y^2}
\\&=
\int_a^b \diff x \int_{\sqrt{1 - x^2}}^\infty \frac{\diff y}{y^2}
\\&=
\int_a^b \frac{\diff x}{\sqrt{1 - x^2}}
\\&\underset{x = \cos\theta}{=}
\int_{\pi - \alpha}^\beta - \frac{\sin \theta}{\sin \theta} \diff \theta = \pi - \alpha - \beta - \cancelto{0}{\gamma} \text{.}
\end{align*}

In the other case, consider a triangle $\Delta = \mrm{ABC}$ with respective angles $\alpha,\beta,\gamma$. Continue the geodesic segment $\mrm{AB}$ to get an intersection $\mrm{D}$ with the boundary.
Let $\Delta' = \mrm{CBD}$ and $\Delta'' = \mrm{ABD}$.

%TODO add figure

Now, $\Delta'$ and $\Delta''$ have a vertex at infinity, so
\begin{align*}
\mu\prs{\Delta} &= \mu\prs{\Delta''} - \mu\prs{\Delta'}
\\&=
\pi - \prs{\alpha + \gamma + \theta} - \prs{\pi - \theta - \prs{\pi - \beta}}
\\&= \pi - \alpha - \beta - \gamma \text{.}
\end{align*}
\end{proof}

\section{Hyperbolic Geometry}

\begin{theorem}
Let $\Delta$ be a hyperbolic triangle with sides of hyperbolic lengths $a,b,c$ and opposite angles $\alpha,\beta,\gamma$.
Assume that $\alpha,\beta,\gamma > 0$ (so there is no vertex at the boundary).

The following holds.

\begin{description}
\item[The Sine Rule:]
\[\frac{\sinh\prs{a}}{\sin \alpha} = \frac{\sinh\prs{b}}{\sin \beta} = \frac{\sinh\prs{c}}{\sin \gamma}\]
\item[The First Cosine Rule:]
\[\cosh\prs{c} = \cosh\prs{a} \cosh\prs{b} - \cos\gamma \sinh\prs{a} \sinh\prs{b}\]
\item[The Second Cosine Rule:]
\[\cosh\prs{c} = \frac{\cos\alpha\cos\beta + \cos\gamma}{\sin \alpha \sin \beta} \text{.}\]
\end{description}
\end{theorem}

\begin{proof}
\begin{description}
\item[The First Cosine Rule:]
We use the disc model to prove the rule. Let $\Delta$ be a triangle in $\mbb{U}$ with sides $a,b,c$ and let $v_a, v_b, v_c$ be the vertices opposite to the respective sides.

We can assume $v_c = 0$ and $v_a = r \in \prs{0,1}$, and denote $v_b = z \in \mbb{U}$.
%TODO add fig 1.4
We have
\[\sinh^2 \prs{\frac{1}{2} \rho_u\prs{z,w}} = \frac{\abs{z-w}}{\prs{1-\abs{z}^2} \prs{1-\abs{w}^2}} \text{,}\]
but
\[\sinh^2\prs{\alpha} = \frac{1}{2} \cosh\prs{2 \alpha} - \frac{1}{2} \alpha\]
because
\begin{align*}
\prs{\frac{e^{\alpha} - e^{-\alpha}}{2}}^\alpha
&= \frac{e^{2 \alpha} - 2 + e^{-2 \alpha}}{4}
\\&= \frac{1}{2} \cdot \frac{e^{2 \alpha} + e^{-2 \alpha}}{2} - \frac{1}{2} \text{.}
\end{align*}
Hence
\[\cosh\prs{\rho_u\prs{z,w}} = \frac{2 \abs{z-w}}{\prs{1-\abs{z}^2} \prs{1-\abs{w}^2}} + 1 \text{.}\]
Then
\begin{align*}
\cosh\prs{a} &= \frac{1 + \abs{z}^2}{1 - \abs{z}^2} \\
\cosh\prs{b} &= \frac{1+r^2}{1-r^2} \\
\cosh\prs{c} &= \frac{2 \abs{z-r}^2}{\prs{1-\abs{z}^2}\prs{1-r^2}} + 1 \text{.}
\end{align*}

Using
\begin{align*}
\sinh\prs{a} &= \sqrt{\cosh^2\prs{\abs{z}} - 1} = \frac{2 \abs{z}}{1 - \abs{z}^2} \\
\sinh\prs{b} &= \sqrt{\cosh^2\prs{r} - 1} = \frac{2r}{1-r^2}
\end{align*}
and the Euclidean cosine rule
\[\cos \gamma = \frac{r^2 + \abs{z}^2 - \abs{2-r}^2}{2r \abs{z}}\]
we get
\begin{align*}
\cosh\prs{a} \cosh\prs{b} - \sinh\prs{a} \sinh\prs{b} \cos\prs{\gamma}
&=
\prs{\frac{1 + \abs{z}^2}{1-\abs{z}^2}} \prs{\frac{1 + r^2}{1-r^2}} - \frac{4r \abs{z}}{\prs{1-r^2}\prs{1-\abs{z}^2}} \cdot \frac{r^2 + \abs{z}^2 - \abs{z-r}^2}{2 r \abs{z}}
\\&=
\frac{\prs{1+r^2}\prs{1+\abs{z}^2} - 2r^2 - 2\abs{z}^2 + 2\abs{z-r}^2}{\prs{1-r^2}\prs{1-\abs{z}^2}}
\\&=
1 + \frac{2 \abs{z-r}^2}{\prs{1-r}^2 \prs{1-\abs{z}^2}}
\\&=
\cosh\prs{c} \text{.}
\end{align*}

\item[The Sine Rule:]
It holds by the first cosine rule that that
\begin{align*}
\prs{\frac{\sinh c}{\sin \gamma}}^2 &=
\frac{\sinh^2 c}{1 - \prs{\frac{\cosh a \cosh b - \cosh c}{\sinh \prs{a} \sinh\prs{b}}}^2}
\\&=
\prs{\cosh^2\prs{a} - 1}\prs{\cosh^2\prs{b} - 1} - \prs{\cosh\prs{a} \cosh\prs{b} - \cosh\prs{c}}^2
\\&= 1 + 2 \cosh\prs{a} \cosh\prs{b} \cosh\prs{c} - \cosh^2\prs{a} - \cosh^2\prs{b} - \cosh^2\prs{c}
\end{align*}
where the last term is symmetric in $a,b,c$.
\end{description}
\end{proof}

\chapter{Fuchsian Groups}

\section{Fuchsian Groups}

\subsection{Definitions}

\begin{definition}[$\mrm{SL}_2\prs{\mbb{R}}$]
Let $\mrm{SL}_2\prs{\mbb{R}}$ be the group of $2 \times 2$ real matrices with determinant $1$, with the topology from $\mbb{R}^4$ via $\pmat{a & b \\ c & d} \mapsto \pmat{a \\ b \\ c \\ d}$.
\end{definition}

Throughout the course, we endow $\mrm{PSL}_2\prs{\mbb{R}}$ with the quotient topology from $\mrm{GL}_2\prs{\mbb{R}}$.

We endow $\mrm{Isom}\prs{\mbb{H}}$ with the following topology.
Let $\tau \in \mrm{Isom}\prs{\mbb{H}} \setminus \mrm{PSL}_2\prs{\mbb{R}}$ and.
$U \subseteq \mrm{Isom}\prs{\mbb{H}}$ is open if and only if $U \cap \mrm{PSL}_2\prs{\mbb{R}}$ and $\tau U \cap \mrm{PSL}_2\prs{\mbb{R}}$ are open.

\begin{exercise}
\begin{enumerate}
\item
$\mrm{SL}_2\prs{\mbb{R}}, \mrm{PSL}_2\prs{\mbb{R}}, \mrm{Isom}\prs{\mbb{H}}$ are topological groups.
\item The actions of $\mrm{PSL}_2\prs{\mbb{R}}$ on $\mbb{H}$ and $\mbb{R} \cup \set{\infty}$ are continuous.
\end{enumerate}
\end{exercise}

\begin{definition}
Let
\[S\mbb{H} \ceq \set{\prs{z,\alpha}}{z \in \mbb{H}, \alpha \in \mbb{C}, \abs{\alpha} = \Im\prs{z}}\]
be the unit tangent bundle of $\mbb{H}$, which is homeomorphic to $\mbb{H} \times S^1$.
\end{definition}

\begin{definition}
For every $g \in \mrm{PSL}_2\prs{\mbb{R}}$ and $\prs{W,\alpha} \in S \mbb{H}$, denote $T_g \cdot \prs{w, \alpha} = \prs{T_g\prs{w}, D\prs{T_g}\prs{w}}$.
\end{definition}

\begin{definition}[Sharply Transitive Action]
A group action is called \emph{sharply transitive} if its transitive and the stabiliser of every element is trivial.
\end{definition}

\begin{lemma}
\begin{enumerate}
\item The map $\mrm{PSL}_2\prs{\mbb{R}} \times S\mbb{H} \to S\mbb{H}$ is a group action.
\item $\mrm{PSL}_2\prs{\mbb{R}}$ acts sharply transitive on $S\mbb{H}$.
\item The map $g \mapsto T_g\prs{\prs{i,i}}$ is a homeomorphism of $\mrm{PSL}_2\prs{\mbb{R}}$ and $S\mbb{H}$.
\end{enumerate}
\end{lemma}

\begin{proof}
\begin{enumerate}
\item Let $\prs{w,\alpha} \in S\mbb{H}$ and $g,h \in \mrm{PSL}_2\prs{\mbb{R}}$. We first show that $g \cdot \prs{w,\alpha} \in S\mbb{H}$.
It holds that
\begin{align*}
\Im\prs{T_g\prs{w}} &=
\abs{\frac{\diff T_g}{\diff z}\prs{w}} \cdot \Im\prs{w} \text{,}
\end{align*}
so
\begin{align*}
\abs{\left. D T_g \right|_{w \prs{\alpha}}} = \abs{\Im\prs{T_g\prs{w}}} \text{.}
\end{align*}

We now have to check that this is an action. It holds that
\begin{align*}
\prs{gh} \cdot \prs{w,\alpha} &= \prs{T_{gh}\prs{\alpha}, D T_{gh} \prs{w} \alpha}
\\&= \prs{T_g \prs{T_h\prs{w}}, D T_g \prs{T_h\prs{w}} \alpha}
\\&= g \cdot \prs{h \cdot \prs{w,\alpha}} \text{.}
\end{align*}

\item Let $\prs{w, \alpha} \in S \mbb{H}$. It is enough to show that there exists a unique $g \in \mrm{PSL}_2\prs{\mbb{R}}$ such that $g \cdot \prs{i,i} = \prs{w, \alpha}$. Recall that geodesic lines in $\mbb{H}$ are oriented generalised semicircles orthogonal to the real axis. Hence there exists a unique geodesic $\ell \colon \mbb{R} \to \mbb{H}$ which passes through $w$ and whose derivative at $w$ is $\alpha$.
Let $\gamma \colon \mbb{R} \to \mbb{H}$ be the geodesic given by $\gamma\prs{t} = ie^{t}$. Since $T_g$ sends geodesics to geodesics, it must send $i$ to $w$, send $\gamma$ to $\ell$, and respect the orientation of $\gamma$ and $\ell$. There exists a unique such $g$.

\item Prove this as an exercise.
\end{enumerate}
\end{proof}

\end{document}