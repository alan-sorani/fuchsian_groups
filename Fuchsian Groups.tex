\documentclass[10pt, twoside]{book}

%%%%%%%%%
% Maths %
%%%%%%%%%

\usepackage{math-fonts}
\usepackage{math-graphics}
\usepackage{math-symbols}
\usepackage{math-theorems}

%%%%%%%%%
% Title %
%%%%%%%%%

\title{Lecture Notes to Fuchsian Groups \\ \large{Winter 2020, Technion}}
\author{Lectures by Chen Meiri \\ \small{Typed by Elad Tzorani}}
\date{\today}

\begin{document}

\maketitle
\tableofcontents

\chapter{Preliminaries}

\section{The Hyperbolic Plane}

\subsection{The Riemann Sphere}

\begin{definition}[The Riemann Sphere]
The \emph{Riemann sphere} is a one-dimensional complex manifold, denoted $\hat{\mbb{C}} = \mbb{C} \cup \set{\infty}$, the charts of which are the following.
\begin{align*}
U_1 &= \prs{\mbb{C}, f_1} \\
U_2 &= \prs{\hat{\mbb{C}} \setminus \set{0}, f_2}
\end{align*}
where
\begin{align*}
f_1 \colon \mbb{C} &\to \mbb{C} \\
z &\mapsto z
\end{align*}
and
\begin{align*}
f_2 \colon \mbb{C} &\to \hat{\mbb{C}} \setminus \set{0} \\
z &\mapsto \frac{1}{z} \text{.}
\end{align*}
\end{definition}

\begin{definition}[Möbius Transformation]
A map $T \colon \hat{\mbb{C}} \to \hat{\mbb{C}}$ of the form
\[z \mapsto \frac{az + b}{cz + d}\]
where $ad - bc \neq 0$ is called a \emph{Möbius transformation}.
\end{definition}

\begin{notation}
\begin{enumerate}
\item We denote the image of $\pmat{a & b \\ c & d} \in \mrm{GL}_2\prs{\mbb{C}}$ in $\mrm{PGL}_2\prs{\mbb{C}}$ by $\bmat{a & b \\ c & d}$.
\item For every $g = \bmat{a & b \\ c & d} \in \mrm{PGL}_2\prs{\mbb{C}}$, we denote by $T_g$ the Möbius transformation $z \mapsto \frac{az + b}{cz + d}$.
\end{enumerate}
\end{notation}

\begin{lemma}
The set of Möbius transformations is a group under composition, and the map $g \mapsto T_g$ is an isomorphism between $\mrm{PGL}_2\prs{\mbb{C}}$ and the group of Möbius transformation.
\end{lemma}

\begin{proof}
It holds that
\begin{align*}
T_{g_1} \circ T_{g_2} \prs{z} &=  \frac{a_1 \prs{\frac{a_2 z + b_2}{c_2 z + d_2}} + 1}{c_1 \prs{\frac{a_2 z + b_2}{c_2 z + d_2}} + d_1}
\\&= \frac{\prs{a_1 a_2 + b_1 c_2} z + \prs{a_1 b_2 + b_1 d_2}}{\prs{c_1 a_2 + d_1 c_2} z + \prs{c_1 b_2 + d_1 d_2}}
\\&= T_{g_1 g_2}\prs{z} \text{.}
\end{align*}
In particular, $T_{g^{-1}}$ is the inverse of $T_g$.

The rest of the proof is clear.
\end{proof}

\begin{definition}[Generalised Circle]
A generalised circle in $\mbb{C}$ is either an Euclidean circle or an Euclidean straight line.
\end{definition}

\begin{lemma}
Let $T \colon \hat{\mbb{C}} \to \hat{\mbb{C}}$ be a Möbius transformation. Then
\begin{enumerate}
\item $T$ is an endomorphism of $\hat{\mbb{C}}$.
\item $T$ is conformal.
\item $T$ sends generalised circles to generalised circles.
\end{enumerate}
\end{lemma}

\subsection{Models of the Hyperbolic Plane}

\begin{definition}[The Upper Half Plane Model for the Hyperbolic Plane]
\begin{enumerate}
\item As a set, define $\mbb{H} \ceq \set{z \in \mbb{C}}{\Im\prs{z} > 0}$.
\item Let $\gamma \colon \brs{0,1} \to \mbb{H}$ be a piecewise continuously differentiable path given by $\gamma\prs{t} = x\prs{t} + iy\prs{t}$ for real functions $x\prs{t}, y\prs{t}$.
The \emph{hyperbolic length} of $\gamma$ is given by
\[h\prs{\gamma} \ceq \int_0^1 \frac{\sqrt{\prs{\frac{\diff x}{\diff t}}^2 + \prs{\frac{\diff y}{\diff t}}^2}}{y\prs{t}} \diff t = \int_0^1 \frac{\abs{\frac{\diff \gamma}{\diff t}}}{y\prs{t}} \diff t \text{.}\]
\item The \emph{hyperbolic distance} $\rho\prs{z,w}$ between two points $z,w \in \mbb{H}$ is defined as $\inf_\gamma h\prs{\gamma}$ where the infimum is taken over all piecewise continuously differentiable paths $\gamma$ from $z$ to $w$.
\end{enumerate}
\end{definition}

\begin{remark}
$\mbb{H}$ is a Riemann surface where for every $z \in \mbb{H}$, the inner product of $T_z H$ is given by
\[\prs{\prs{x_1, y_1},\prs{x_2,y_2}} = \frac{x_1 x_2 + y_1 y_2}{\prs{\Im z}^2} \text{.}\]
In particular, Euclidean angles are equal to hyperbolic angles.
\end{remark}

\begin{definition}[The Disc Model for the Hyperbolic Plane]
\begin{enumerate}
\item As a set, define $\mbb{U} \ceq \set{z \in \mbb{C}}{\abs{z} < 1}$.
\item Let $\gamma \colon \brs{0,1} \to \mbb{U}$ be a piecewise continuously differentiable path. The \emph{hyperbolic length} of $\gamma$ is given by
\[h_u\prs{\gamma} \ceq \int_0^1 \frac{2 \abs{\frac{\diff \gamma}{\diff t}}}{1 - \abs{\gamma\prs{t}}^2} \diff t \text{.}\]

\item The \emph{hyperbolic distance} $\rho_u\prs{z,w}$ between $z,w \in \mbb{U}$ is defined to be $\inf_\gamma h\prs{\gamma}$ where the infimum is taken over all piecewise continuously differentiable paths from $z$ to $w$.
\end{enumerate}
\end{definition}

\begin{remark}
It is clear that hyperbolic circles around $0$ are exactly Euclidean circles around it (with a generally different radius).
\end{remark}

\begin{remark}
Rotations around $0$ are isometries in the disc model.
\end{remark}

\begin{lemma}
Let $\pi$ be the Möbius transformation defined by
\[\pi\prs{z} = \frac{iz + 1}{z + i} \text{.}\]
Then
\begin{enumerate}
\item $\pi$ is a bijection from $\mbb{H}$ to $\mbb{U}$.
\item For every piecewise continuously differentiable path $\gamma \colon \brs{0,1} \to \mbb{H}$,it holds that $h_u\prs{\pi\prs{\gamma}} = h\prs{\gamma}$.
In particular, $\pi$ is an isometry.
\end{enumerate}
\end{lemma}

\begin{proof}
\begin{enumerate}
\item It holds that
\begin{align*}
\pi\prs{-1} &= -1 \\
\pi\prs{0} &= -i \\
\pi\prs{1} = 1 \text{.}
\end{align*}
Since Möbius transformations send generalised circles to generalised circles we get that $\pi$ sends $\mbb{R}$ to the unit circle.
Since $\pi\prs{i} = 0$ and $\pi$ is a homeomorphism of the Riemann sphere, we get the result.

\item Let $\gamma \colon \brs{0,1} \to \mbb{H}$ be a piecewise continuously differentiable path. Denote $\psi = \pi^{-1}$ and $\delta = \pi\prs{\gamma}$.
Then
\[\psi\prs{z} = \frac{iz - 1}{-z + i} = \frac{\prs{iz - 1}{- \bar{z} - i}}{\prs{-z + i}{-\bar{z} - i}} = \frac{\prs{z + \bar{z}} + i \prs{1 - \abs{z}^2}}{\abs{-z + i}^2} \text{.}\]
So,
\[\Im\prs{\psi\prs{z}} = \frac{1 - \abs{z}^2}{\abs{-z + i}^2} \text{.}\]
Since
\[\frac{\diff \psi}{\diff z} = \frac{-2}{\prs{-z+i}^2} \text{,}\]
we get that
\begin{align*}
h\prs{\gamma} &= \int_0^1 \frac{\abs{\frac{\diff \gamma}{\diff t}}}{\Im\prs{\gamma\prs{t}}} \diff t
\\&=
\int_0^1 \frac{\abs{\frac{\diff \psi\prs{\delta}}{\diff t}}}{\Im \prs{\psi\prs{\delta\prs{t}}}} \diff t
\\&=
\int_0^1 \frac{\abs{\frac{\diff \psi}{\diff z} \prs{\delta\prs{t}} \frac{\diff \delta}{\diff t}}}{\Im \prs{\psi\prs{\delta\prs{t}}}}
\\&= \int_0^1 \frac{2 \abs{\frac{\diff \delta}{\diff t}}}{1 - \abs{\delta\prs{t}}^2} \diff t
\\&= h_u\prs{\delta} \text{.}
\end{align*}

\end{enumerate}
\end{proof}

\subsection{Isometries of the Hyperbolic Plane}

\begin{lemma}
For every $g \in \bmat{a & b \\ c & d} \in \mrm{PSL}_2\prs{\mbb{R}}$ it holds that
\[T_g\prs{\mbb{H}} \subseteq \mbb{H} \text{.}\]
\end{lemma}

\begin{proof}
It's enough to show the inclusion $T_g\prs{\mbb{H}} \subseteq \mbb{H}$ since then
\[T_{g^{-1}}\prs{\mbb{H}} = \prs{T_g}^{-1} \prs{\mbb{H}} \subseteq \mbb{H}\]
which implies $T_g\prs{\mbb{H}} \supseteq \mbb{H}$ by applying $T_g$.

Now, we have
\begin{align*}
T_g\prs{z} &= \frac{az + b}{cz + d}
\\&=
\frac{\prs{az + b}\prs{c \bar{z} + d}}{\abs{cz + d}^2}
\\&=
\frac{ac \abs{z}^2 + adz + bc \bar{z} + bg}{\abs{cz + d}^2} \text{.}
\end{align*}
Thus,
\begin{align*}
\Im\prs{T_g\prs{z}} &= \frac{T_g\prs{z} - \overline{T_g\prs{z}}}{2 i}
\\&=
\frac{\prs{ad - bc} z - \prs{ad - bc} \bar{z}}{2i \abs{cz + d}^2}
\\&\underset{ad - bc = 1}{=}
\frac{\Im\prs{z}}{\abs{cz + d}^2} \text{.}
\end{align*}

\end{proof}

This lemma allows us to identify $\mrm{PSL}_2\prs{\mbb{R}}$ as a subgroup of $\mrm{Sym}\prs{\mbb{H}}$. The next lemma shows that even more is true.

\begin{lemma}
$\mrm{PSL}_2\prs{\mbb{R}} \subseteq \mrm{Isom}\prs{\mbb{H}}$.
\end{lemma}

\begin{proof}
It's enough to show that for every $g \in \mrm{PSL}_2\prs{\mbb{R}}$ and every piecewise continuously differentiable path $\gamma$ it holds that $h\prs{\gamma} = h\prs{T_g\prs{\gamma}}$.
Denote $T = T_g$ and $\delta = T\prs{\gamma}$. Then
\begin{align*}
h\prs{\delta} &= \int_0^1 \frac{\abs{\frac{\diff \delta}{\diff t}}}{\Im\prs{\delta\prs{t}}} \diff t
\\&=
\int_0^1 \frac{\abs{\frac{\diff T}{\diff z} \prs{\gamma\prs{t}} \frac{\diff \gamma}{\diff t}}}{\Im\prs{\delta\prs{t}}} \diff t
\\&\underset{\star}{=}
\int_0^1 \frac{\abs{\frac{\diff \gamma}{\diff t}}}{\gamma\prs{t}} \diff t
\\&=
h\prs{\gamma}
\end{align*}
where $\star$ follows from
\begin{align*}
\Im\prs{T_g\prs{z}} &= \frac{\Im \prs{z}}{\abs{cz + d}^2} \oplus \frac{\diff T}{\diff z} \\&=
\frac{a\prs{cz + d} - c\prs{az + b}}{\prs{cz + d}^2}
\\&= \frac{1}{\prs{cz + d}^2} \text{.}
\end{align*}
\end{proof}

\begin{corollary}
$\mrm{Isom}\prs{\mbb{H}}$ acts transitively on $\mbb{H}$.
\end{corollary}

\begin{proof}
It's enough to show that for every $z \in \mbb{H}$ there's $g \in \mrm{PSL}_2\prs{\mbb{R}}$ such that $T_g\prs{z} = i$.

If $z = x + yi$, take $g = \pmat{\frac{1}{\sqrt{y}} & - \frac{x}{\sqrt{y}} \\ 0 & \frac{1}{\sqrt{y}}}$, then
\[T_g\prs{z} = \frac{1}{y} \prs{x + yi} - \frac{x}{y} = i \text{.}\]
\end{proof}

\begin{lemma}
Let $\pi \colon \mbb{H} \to \mbb{U}$ be the isometry $z \mapsto \frac{iz + 1}{z + i}$ which we defined previously.
Then
\[\set{\pi T_g \pi^{-1}}{g \in \mrm{PSL}_2\prs{\mbb{R}}} = \set{\pmat{r & s \\ \bar{r} & \bar{s}}}{\substack{r,s \in \mbb{C} \\ \abs{r}^2 - \abs{s}^2 = 1}} \text{.}\]
In particular, by taking $r = e^{i \theta}$ and $s = 0$ we see that the action of $\mrm{PSL}_2\prs{\mbb{R}}$ on $\mbb{U}$ contains all the rotations around $0$.
\end{lemma}

\begin{proof}
It holds that
\[\pmat{i & 1 \\ 1 & i} \pmat{a & b \\ c & d} \pmat{i & -1 \\ -1 & i} = \frac{1}{2} \pmat{\prs{a+d} + i\prs{b-c} & b + c + i\prs{a-d} \\ \prs{b+c} - i \prs{a+d} & \prs{a+d} - i\prs{b-c}} \text{.}\]
Now, $\prs{a+d, a-d, b+c, b-c}$ can be any $4$-tuple. Specifically, for every $r,s \in \mbb{C}$ we have $a,b,c,d \in \mbb{R}$ such that $\pi T_g \pi^{-1} = \pmat{r & s \\ \bar{s} & \bar{r}}$, and by the equality from the determinants we get that $\bmat{a & b \\ c & d } \in \mrm{PSL}_2\prs{\mbb{R}}$.
\end{proof}

\begin{corollary}
Let $z_1, z_2, w_1, w_2 \in \mbb{H}$ be such that $\rho\prs{z_1, w_1} = \rho\prs{z_2, w_2}$, then there exists $g \in \mrm{PSL}_2\prs{\mbb{R}}$ such that $T_g\prs{z_1} = z_2$ and $T_g\prs{w_1} = w_2$.
\end{corollary}

\begin{proof}
Since $\mrm{PSL}_2\prs{\mbb{R}}$ acts transitively on $\mbb{H}$ we can assume that $z_1 = z_2$ and show that $\mrm{Stab}\prs{z_1}$ acts transitively on $\set{w \in \mbb{H}}{\rho\prs{z_1, w} = \rho\prs{z_1, w_1}}$.
We already showed this in the disc model, in the case $z_1 = i$.
\end{proof}

\begin{definition}
Let $\prs{X,d}$ be a metric space.
\begin{enumerate}
\item Let $x,y \in X$. A path $\gamma \colon \brs{a,b} \to X$ which joins $x$ and $y$ is called a \emph{geodesic segment} if for every $a \leq t_1 \leq t_2 \leq b$ it holds that $\abs{t_2 - t_1} = d\prs{\gamma\prs{t_1}, \gamma\prs{t_2}}$.

\item A path $\gamma \colon \mbb{R} \to X$ is called a \emph{geodesic line} if for every $a < b$ it holds that $\left. \gamma \right|_{\brs{a,b}}$ is a geodesic segment.
\end{enumerate}
\end{definition}

\begin{remark}
Let $\gamma$ be a geodesic segment or line.
Then $\gamma$ is determined by the image of $\gamma$ up to a composition with an isometry of $\mrm{R}$. Thus, we can identify geodesic segments and lines with their image up to orientation.
\end{remark}

\begin{lemma}
Let $b > a > 0$ be real numbers. Then $\set{iy}{a \leq y \leq b}$ is the unique geodesic segment between $ia$ and $ib$ and $\set{iy}{y > 0}$ is the unique geodesic line through $ia$ and $ib$.
\end{lemma}

\begin{proof}
We begin with the first part of the lemma.
Let $\gamma \colon \brs{0,1} \to \mbb{H}$ be a piecewise continuously differentiable path joining $ia$ and $ib$. For $t \in \brs{0,1}$ denote $\gamma\prs{t} = x\prs{t} + iy\prs{t}$ where $x\prs{t},y\prs{t} \in \mbb{R}$.
Then
\begin{align*}
h\prs{\gamma} &= \int_0^1 \frac{\sqrt{\prs{\frac{\diff x}{\diff t}}^2 + \prs{\frac{\diff y}{\diff t}}^2}}{y\prs{t}} \diff t
\\&\underset{\star}{\geq}
\int_0^1 \frac{\abs{\frac{\diff y}{\diff t}}}{y\prs{t}} \diff t
\\&\geq
\int_0^1 \frac{\frac{\diff y}{\diff t}}{y\prs{t}} \diff t
\\&=
\ln\prs{\frac{b}{a}} \text{.}
\end{align*}
Thus, $\rho\prs{ia, ib} \geq \ln\prs{\frac{b}{a}}$. If $y\prs{t} = i\prs{\prs{b-a}t + a}$, the above inequalities are equalities so $\rho\prs{ia, ib} = \ln\prs{\frac{b}{a}}$. The inequality $\star$ is an equality if and only if $x\prs{t} = 0$ for all $t \in \brs{a,b}$. It follows that the unique geodesic segment between $a$ and $b$ is $\set{iy}{a \leq y \leq b}$.

Now, it is clear that $\set{iy}{y > 0}$ is a geodesic line which passes through $ia$ and $ib$.
We want to show it's unique.

Assume towards a contradiction that there exists a geodesic line $\ell$ between $ia$ and $ib$ which isn't the positive part of the $y$-axis.
Then there's $z = x + iy \in \ell$ for which $x \neq 0$ and $\rho\prs{z, ia} > \rho\prs{z, ib}$. By the previous lemma, there exists $g \in \mrm{PSL}_2\prs{\mbb{R}}$ such that $T_g\prs{ia} = ia$ and $T_g\prs{z} \in i \mbb{R}$. Since $T_g$ sends generalised circles to generalised circles, $T_g\prs{ib} \notin i \mbb{R}$. Indeed, otherwise the image of the segment between $ia$ and $ib$ would belong to $i \mbb{R}$, and since $T_g$ sends generalised circles to generalised circles, it would send $i\mbb{R}$ to itself.

We get that there exists a geodesic between $ia$ and $T_g\prs{z} = ic$ which is not contained in $i\mbb{R}$, and this is impossible.
\end{proof}

\begin{theorem}

\begin{enumerate}
\item
Every distinct points $z,w \in \mbb{H}$ are contained in a unique geodesic segment and a unique geodesic line.

\item The geodesics in $\mbb{H}$ are semicircles and lines orthogonal to the real axis.
\end{enumerate}
\end{theorem}

\begin{proof}
\begin{enumerate}
\item
For every $g \in \mrm{PSL}_2\prs{\mbb{R}}$ it holds that $T_g\prs{\mbb{R} \cup \set{\infty}} = \mbb{R} \cup \set{\infty}$. If $z,w \in \mbb{H}$, by a previous lemma there exists $g \in \mrm{PSL}_2\prs{\mbb{R}}$ such that $T_g\prs{z} = ia$ and $T_g\prs{w} = ib$ for some $a,b \in \mbb{R}_+$.
Thus, $T_g^{-1} \prs{\brs{ia, ib}}$ is the unique geodesic segment between $z$ and $w$.

\item This follows from the fact that Möbius circles are conformal, send generalised circles to generalised circles, and sends $\mbb{R} \cup \set{\infty}$ to itself.      
\end{enumerate}
\end{proof}

\begin{corollary}
The geodesic segment in $\mbb{U}$ are segments of straight lines through zero or arcs of circles which are orthogonal to the unit circles.
\end{corollary}

\begin{theorem}
Let $z,w \in \mbb{H}$. Then
\begin{align*}
\sinh\prs{\frac{1}{2} \rho\prs{z,w}} = \frac{\abs{z-w}}{2 \prs {\Im\prs{z} \Im\prs{w}}^{\frac{1}{2}}} \text{.}
\end{align*}

\begin{proof}
Since $\mrm{PSL}_2\prs{\mbb{R}} \subseteq \mrm{Isom}\prs{\mbb{H}}$, the left side of the equation is invariant under the action of $\mrm{PSL}_2\prs{\mbb{R}}$. We first show that the right side is also invariant.

It's clear that the right side is invariant under maps of the form $z \mapsto az + b$ for $a,b \in \mbb{R}$. Since $\mrm{PSL}_2\prs{\mbb{R}}$ (viewed as a group of Möbius transformations) is generated by maps of the forms
\begin{align*}
z &\mapsto az + b, \, a,b\in\mbb{R} \\
z &\mapsto - \frac{1}{z}
\end{align*}
it's enough to show that the right side is invariant under these maps.

The right side is indeed invariant under $\frac{1}{z}$ since
\begin{align*}
\frac{\abs{\frac{1}{z} - \frac{1}{w}}}{2 \prs{\Im\prs{\frac{1}{z}} \Im\prs{\frac{1}{w}}}^{\frac{1}{2}}}
&=
\frac{\abs{\frac{z-w}{zw}}}{2 \prs{\Im\prs{\frac{z}{\abs{z}^2}} \Im\prs{\frac{w}{\abs{w}^2}}}^{\frac{1}{2}}}
\\&=
\frac{\abs{z-w}}{2 \prs{\Im\prs{z}\Im\prs{w}}^{\frac{1}{2}}} \text{.}
\end{align*}

Since both sides of the equation are invariant under the action of $\mrm{PSL}_2\prs{\mbb{R}}$, it's enough to prove the equality for $z=i$ and $w = ir$ for some $r \in \mbb{R}_+$.
Indeed,
\begin{align*}
\sinh\prs{\frac{1}{2} \rho\prs{i, ir}} &=
\sinh\prs{\frac{1}{2} \abs{\ln r}}
\\&=
\frac{\abs{\sqrt{r} - \frac{1}{\sqrt{r}}}}{2}
\\&=
\frac{\abs{r-1}}{2 \sqrt{r}}
\\&=
\frac{\abs{i - ir}}{2 \prs{\Im\prs{i} \Im\prs{ir}}^{\frac{1}{2}}} \text{.}
\end{align*}

\end{proof}
\end{theorem}

\end{document}